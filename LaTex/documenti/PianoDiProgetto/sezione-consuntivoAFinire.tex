\section{Consuntivo parziale}\label{consuntivi}
In questa sezione vengono riportate le spese effettivamente sostenute. Verrà mostrato l'effettivo consumo di ore sia per ruolo che per persona ed in base al risultato avremo un bilancio:
\begin{itemize}
	\item positivo: se il consuntivo è inferiore al preventivo;
	\item negativo: se il consuntivo è maggiore al preventivo;
	\item in pari: se il consuntivo rispecchia a pieno il preventivo.
\end{itemize}
I valori positivi indicano un eccesso di ore, i negativi un consumo inferiore a quello preventivato.

\subsection{Analisi dei Requisiti}
La tabella sottostante riporta la differenza tra preventivo e consuntivo del periodo di \AR{} divisa per ruolo.
\begin{table}[H]
	\centering
	\begin{tabular}{|c|c|c|}
		\hline
		\textbf{Ruolo} &
		\textbf{Ore} &
		\textbf{Costo} \\
		\hline
		\Responsabile & -1 & -30\\
		\hline
		\Amministratore & 0 & 0\\
		\hline
		\Analista & 2 & 50\\
		\hline
		\Progettista & 0 & 0 \\
		\hline
		\Verificatore & 0 & 0\\
		\hline
		\Programmatore & 0 & 0 \\
		\hline
		\textbf{Totale} & \textbf{1} & \textbf{20} \\
		\hline
	\end{tabular}
	\caption{Differenza preventivo consuntivo per ruolo, periodo di \AR}
\end{table}

La tabella sottostante riporta la differenza tra preventivo e consuntivo della periodo di \AR{} divisa per componente.
\begin{table}[H]
	\centering
	\begin{tabular}{|l|c|c|c|c|c|c|c|}
		\hline
		\textbf{Nominativo} & 
		\multicolumn{6}{c|}{\textbf{Ore per ruolo}} & 
		\textbf{Ore totali} \\
		& Re & Am & An & Pj & Pr & Ve & \\
		\hline
		Nicola Dal Maso & & &1 & & & & 1 \\
		Lorenzo Ferrarin & & &1.5 & & & & 1.5 \\
		Beatrice Guerra & -2 & & & & & & -2 \\
		Marco Ponchia & & & & & & & 0 \\
		Tommaso Rosso & & & & & & & 0 \\
		Alice V. Sasso & & & & & & & 0 \\
		Mattia Zecchinato & 1& &-0.5 & & & & .5  \\
		\hline
	\end{tabular}
	\caption{Differenza preventivo consuntivo per componente, periodo di \AR}
\end{table}
\subsubsection{Conclusioni}
Per completare il periodo di \AR{} è stata necessaria un'ora di lavoro in più di quanto preventivato, con un aumento di spesa di \textbf{20€}.
\subsubsection{Impatto sul preventivo a finire}
Non essendo il periodo di \AR{} inclusa nella proposta, non ci saranno ripercussioni nel preventivo. Per quanto riguarda il preventivo totale, comprendente le ore non rendicontate, lo scostamento rilevato non è influente poiché corrisponde a meno di un'ora di lavoro.

\subsection{Analisi di Dettaglio}
La tabella sottostante riporta la differenza tra preventivo e consuntivo del periodo di \AD{} divisa per ruolo.
\begin{table}[H]
	\centering
	\begin{tabular}{|c|c|c|}
		\hline
		\textbf{Ruolo} &
		\textbf{Ore} &
		\textbf{Costo} \\
		\hline
		\Responsabile & 0 & 0\\
		\hline
		\Amministratore & 0 & 0\\
		\hline
		\Analista & 5 & 125\\
		\hline
		\Progettista & 0 & 0 \\
		\hline
		\Verificatore & 0 & 0\\
		\hline
		\Programmatore & 0 & 0 \\
		\hline
		\textbf{Totale} & \textbf{5} & \textbf{125} \\
		\hline
	\end{tabular}
	\caption{Differenza preventivo consuntivo per ruolo, periodo di \AD}
\end{table}

La tabella sottostante riporta la differenza tra preventivo e consuntivo della periodo di \AD{} divisa per componente.
\begin{table}[H]
	\centering
	\begin{tabular}{|l|c|c|c|c|c|c|c|}
		\hline
		\textbf{Nominativo} & 
		\multicolumn{6}{c|}{\textbf{Ore per ruolo}} & 
		\textbf{Ore totali} \\
		& Re & Am & An & Pj & Pr & Ve & \\
		\hline
		Nicola Dal Maso & & & & & & & 0 \\
		Lorenzo Ferrarin & & & & & & & 0 \\
		Beatrice Guerra & & &2 & & & & 2 \\
		Marco Ponchia & & & & & & & 0 \\
		Tommaso Rosso & & & & & & & 0 \\
		Alice V. Sasso & & & 2& & & & 2 \\
		Mattia Zecchinato & & &1 & & & & 1  \\
		\hline
	\end{tabular}
	\caption{Differenza preventivo consuntivo per componente, periodo di \AD}
\end{table}
\subsubsection{Conclusioni}
Per completare il periodo di \AD{} è stato necessario un investimento non preventivato di cinque ore, che ha portato ad un aumento dei costi di \textbf{125€}. Lo sforamento dal preventivo è stato causato da una visione ottimistica dei tempi e allo studio di alcuni punti risultati carenti dopo il periodo di \AR{}, sono stati quindi analizzati i periodi successivi e risultano realistici per la realizzazione del progetto.
\subsubsection{Impatto sul preventivo a finire}
Questa fase vista la sua durata ha avuto un impatto relativamente importante sul preventivo a finire, a causa delle modifiche che si sono rese necessarie.
È stata consumata una buona parte del margine previsto sul preventivo iniziale, che ci obbligherà a organizzare gli impegni per ruolo dei membri del gruppo in maniera più efficiente per le fasi successive.

\subsection{Progettazione Architetturale}
La tabella sottostante riporta la differenza tra preventivo e consuntivo del periodo di \PA{} divisa per ruolo.
\begin{table}[H]
	\centering
	\begin{tabular}{|c|c|c|}
		\hline
		\textbf{Ruolo} &
		\textbf{Ore} &
		\textbf{Costo} \\
		\hline
		\Responsabile & 0 & 0\\
		\hline
		\Amministratore & 0 & 0\\
		\hline
		\Analista & -2 & -50\\
		\hline
		\Progettista & -2 & -44 \\
		\hline
		\Verificatore & 1 & 15 \\
		\hline
		\Programmatore & 0 & 0 \\
		\hline
		\textbf{Totale} & \textbf{-3} & \textbf{-79} \\
		\hline
	\end{tabular}
	\caption{Differenza preventivo consuntivo per ruolo, periodo di \PA}
\end{table}

La tabella sottostante riporta la differenza tra preventivo e consuntivo della periodo di \PA{} divisa per componente.
\begin{table}[H]
	\centering
	\begin{tabular}{|l|c|c|c|c|c|c|c|}
		\hline
		\textbf{Nominativo} & 
		\multicolumn{6}{c|}{\textbf{Ore per ruolo}} & 
		\textbf{Ore totali} \\
		& Re & Am & An & Pj & Pr & Ve & \\
		\hline
		Nicola Dal Maso & & & & & & & 0 \\
		Lorenzo Ferrarin & & & & & & & 0 \\
		Beatrice Guerra & & & -2& & & 1& -1 \\
		Marco Ponchia & & & & & & & 0 \\
		Tommaso Rosso & & & & & & & 0 \\
		Alice V. Sasso & & & 1& -2& & & -1 \\
		Mattia Zecchinato & & & -1& & & & -1 \\
		\hline
	\end{tabular}
	\caption{Differenza preventivo consuntivo per componente, periodo di \PA}
\end{table}
\subsubsection{Conclusioni}
Durante il periodo di \PA{} sono state necessarie delle piccole riorganizzazioni nell'impegno orario per alcuni componenti, che non hanno mai riguardato un periodo superiore alle due ore a persona.
I cambiamenti sono stati resi necessari dal verificarsi di uno dei rischi preventivati, la rottura di un componente hardware da parte di un membro del gruppo, che non ha quindi portato ulteriori disguidi venendo affrontato con tempestività.
Queste scelte hanno portato ad un risparmio di tre ore e di \textbf{-79€} sul preventivo.
\subsubsection{Impatto sul preventivo a finire}
Il risparmio ottenuto in questo periodo, sia dal punto di vista dei costi, che da quello dell'impegno orario, ha portato ad un riavvicinamento al preventivo iniziale. Il riassestamento delle ore per alcune persone ha permesso di sistemare gli squilibri creatisi nei periodi precedenti evitando così un superamento del limite massimo delle ore. Questo ci permetterà di mantenere le fasi successive così come preventivato, senza apportare ulteriori modifiche.

\subsection{\PD{} e \Cod{}}
All'inizio dei periodi di \PD{} e \Cod{} è stato necessario effettuare una ripianificazione delle ore, data la decisione da parte del gruppo di posticipare la Revisione di Qualifica. Grazie all'esperienza acquisita fin'ora è stato possibile pianificare in modo corretto le attività, incrementando di 15 ore l'impegno in termini di ore/persona complessivo. La tabella seguente riporta la suddivisione di tale impegno per i diversi ruoli:
\begin{table}[H]
	\centering
	\begin{tabular}{|c|c|c|}
		\hline
		\textbf{Ruolo} &
		\textbf{Ore} &
		\textbf{Costo} \\
		\hline
		\Responsabile & 0 & 0\\
		\hline
		\Amministratore & 0 & 0\\
		\hline
		\Analista & 0 & 0\\
		\hline
		\Progettista & 10 & 220 \\
		\hline
		\Verificatore & 2 & 30 \\
		\hline
		\Programmatore & 3 & 45 \\
		\hline
		\textbf{Totale} & \textbf{15} & \textbf{295} \\
		\hline
	\end{tabular}
	\caption{Differenza preventivo consuntivo per ruolo, periodo di \PA}
\end{table}

\subsubsection{Conclusioni}
La scelta di posticipare la partecipazione alla Revisione di Qualifica, ha comportato un incremento delle ore da rendicontare, ma ha permesso al gruppo di lavorare in maniera più ordinata e conforme alle norme ed ottenendo anche una maggiore consapevolezza degli strumenti utilizzati.

\subsubsection{Impatto sul preventivo a finire}
L'aumento delle ore ha comportato un diretto aumento dei costi, che portano il preventivo a finire ad allinearsi al preventivo iniziale presentato durante la gara d'appalto.
La situazione attuale comporterà un'attenzione particolare a possibili variazioni durante la fase di \VV{}.

\subsection{\VV{}}\label{consuntivoVV}
All'inizio dei periodi di \VV{} è stato necessario effettuare una ripianificazione delle ore, data dalla necessità di incrementare il numero di ore dedicate alla programmazione, non per questo però limitando le ore dedicate alla \VV{}, attività principale per questo periodo. La tabella seguente riporta la suddivisione di tale impegno per i diversi ruoli:
\begin{table}[H]
	\centering
	\begin{tabular}{|c|c|c|}
		\hline
		\textbf{Ruolo} &
		\textbf{Ore} &
		\textbf{Costo} \\
		\hline
		\Responsabile & -2 & -60\\
		\hline
		\Amministratore & -4 & -80\\
		\hline
		\Analista & 0 & 0\\
		\hline
		\Progettista & -4 & -88 \\
		\hline
		\Verificatore & 12 & 180 \\
		\hline
		\Programmatore & 12 & 180 \\
		\hline
		\textbf{Totale} & \textbf{14} & \textbf{146} \\
		\hline
	\end{tabular}
	\caption{Differenza preventivo consuntivo per ruolo, periodo di \VV}
\end{table}

La tabella sottostante riporta la differenza tra preventivo e consuntivo della periodo di \VV{} divisa per componente.
\begin{table}[H]
	\centering
	\begin{tabular}{|l|c|c|c|c|c|c|c|}
		\hline
		\textbf{Nominativo} & 
		\multicolumn{6}{c|}{\textbf{Ore per ruolo}} & 
		\textbf{Ore totali} \\
		& Re & Am & An & Pj & Pr & Ve & \\
		\hline
		Nicola Dal Maso & & & &-2 & 4& & 2 \\
		Lorenzo Ferrarin & & & & & & 2& 2 \\
		Beatrice Guerra & & & & &2 & & 2 \\
		Marco Ponchia & & & & & 2& & 2 \\
		Tommaso Rosso & & & & -4& 4&2 & 2 \\
		Alice V. Sasso & & -2& & & & 4& 2 \\
		Mattia Zecchinato &-2 & & & & &4 & 2 \\
		\hline
	\end{tabular}
	\caption{Differenza preventivo consuntivo per componente, periodo di \VV}
\end{table}

\subsubsection{Conclusioni}
In questo periodo finale si è reso necessario un riassestamento delle risorse, in quanto dopo la Revisione di Qualifica abbiamo dovuto apportare alcune modifiche minori e varie migliorie al prodotto, causando un incremento delle ore, principalmente per il ruolo di Programmatore. Questa problematica deriva da un rallentamento dei lavori nel periodo precedente e dalla visione piuttosto ottimistica dell'utilizzo delle risorse per il completamento del progetto.

\subsubsection{Impatto sui costi}
L'incremento delle ore dei Programmatori, nonostante l'adeguamento delle altre risorse, ha comportato una variazione sul preventivo di \textbf{146€}, portando così il costo totale rendicontato a 13585€ e causando un aumento di \textbf{85€}, non previsto dal preventivo iniziale presentato durante l'appalto.

\clearpage

\section{Consuntivo finale}\label{consuntivoFinale}
La tabella sottostante mette in evidenza il totale delle ore rendicontate suddivise per ruolo.
\begin{table}[H]
	\centering
	\begin{tabular}{|c|c|c|}
		\hline
		\textbf{Ruolo} &
		\textbf{Ore} &
		\textbf{Costo} \\
		\hline
		Responsabile & 33 & 990\\
		\hline
		Amministratore & 30 & 600\\
		\hline
		Analista & 27 & 675\\
		\hline
		Progettista & 233 & 5126 \\
		\hline
		Programmatore & 171 & 2565 \\
		\hline
		Verificatore & 241 & 3615\\
		\hline
		\textbf{Totale} & \textbf{735} & \textbf{13585} \\
		\hline
	\end{tabular}
	\caption{Costo totale rendicontato per ruolo}
\end{table}

La tabella sottostante mostra come sono state suddivise le ore nell'arco dell'intero progetto per ogni componente del gruppo.
\begin{table}[H]
	\centering
	\begin{tabular}{|l|c|c|c|c|c|c|c|}
		\hline
		\textbf{Nominativo} & 
		\multicolumn{6}{c|}{\textbf{Ore per ruolo}} & 
		\textbf{Ore totali} \\
		& Re & Am & An & Pj & Pr & Ve & \\
		\hline
		Nicola Dal Maso &8 &5 &3 &36 &30 &23 & 105 \\
		Lorenzo Ferrarin &5 &5 &4 &37 &21 &33 & 105 \\
		Beatrice Guerra &5 &4 &4 &33 &26 &33 & 105 \\
		Marco Ponchia &3 &8 &8 &34 &22 &30 & 105 \\
		Tommaso Rosso &5 &2 &2 &29 &32 &35 & 105 \\
		Alice V. Sasso &4 &4 &4 &30 &14 &49 & 105 \\
		Mattia Zecchinato &3 &4 &2 &32 &26 &38 & 105 \\
		\hline
	\end{tabular}
	\caption{Ore totali rendicontate per componente}
\end{table}

\subsection{Conclusioni}
Al termine del progetto risultano soddisfatti gli obbiettivi preposti, nonostante i risultati finali mostrino un superamento del preventivo presentato durante la gare d'appalto di 85€, portando così il totale rendicontato alla cifra di 13585€. \\
L'organizzazione del gruppo ha portato ad usufruire dell'intero monte ore a disposizione, permettendo ad ognuno di ricoprire ogni ruolo per un tempo abbastanza uniforme agli altri membri del gruppo.\\
Durante il lavoro di gruppo non sono sorte significative problematiche interpersonali e la maggior parte dei problemi presentati sono stati affrontati con tempestività senza creare disguidi. Da notare che la mancanza di esperienza nell'organizzazione del progetto e nel lavoro suddiviso tra i vari membri ci ha portato a dover riadattare più di una volta il piano di lavoro causando modifiche ai costi preventivati.\\
I rischi di inesperienza e sforamento dei costi, presentati durante l'analisi dei rischi, si sono dunque rivelati vere e proprie problematiche, strettamente interconnesse tra loro e decisive per la riuscita del progetto. In particolare la conoscenza iniziale delle tecnologie utilizzate è risultata non sufficiente, richiedendo un gran numero di ore per l'apprendimento personale, ed alcune di esse hanno richiesto un ulteriore sforzo non trascurabile nell'utilizzo, a causa della poca affidabilità (e.g. RocketChat). Questo ha causato vari ritardi sul completamento dei task nei periodi iniziali, che, sommati ad alcuni errori di pianificazione, la quale inizialmente non teneva conto delle sessioni d'esame ad esempio, hanno reso difficoltoso il rispetto delle tempistiche e dei costi inizialmente dichiarati. È stato infatti necessario posticipare la \RQ{}, e di conseguenza la \RA{}, portando da una parte ad una ulteriore sovrapposizione per la maggior parte dei membri del gruppo con la sessione d'esami estiva e con lo stage curricolare per una minoranza, aumentando a 5 gli studenti-lavoratori; dall'altra ad un aumento delle ore di impegno personale, con conseguente aumento dei costi.

Il costo finale spettante al Committente è pertanto di \textbf{13500€}, mentre sono a carico del gruppo fornitore i restanti \textbf{85€}.