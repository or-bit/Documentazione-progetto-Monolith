\letteraGlossario{F}
\definizione{Fattorino}
Il fattorino è il tipo di utente dell’applicazione che sfrutta quest’ultima per visualizzare la lista di consegne da effettuare e confermare quando ne completa una.

\definizione{fork}
Nell'ambito dell'ingegneria del software indica lo sviluppo di un nuovo progetto software che parte dal codice sorgente di un altro già esistente. Quando si verifica un fork entrambe le parti coinvolte nella realizzazione del progetto iniziano il lavoro sulla stessa base di codice.

\definizione{Framework}
Architettura logica di supporto (spesso un'implementazione logica di un particolare \glossario{design pattern}) su cui un software può essere progettato e realizzato, spesso facilitandone lo sviluppo da parte del team di sviluppatori.

\definizione{Front-end}
Il front-end denota la parte visibile all’utente e con cui egli può interagire, ovvero l’interfaccia utente. \`{E} responsabile dell’acquisizione dei dati in ingresso e della loro elaborazione con modalità conformi a specifiche predefinite e invarianti, tali da renderli utilizzabili dal \glossario{back-end}.

\definizione{Full-stack}
Una soluzione (piattaforma o software) stack è un gruppo di sottosistemi o componenti software necessari per creare una piattaforma completa che non richiede software aggiuntivi per supportare le applicazioni. Una soluzione full-stack è dunque una piattaforma completa.	

\definizione{12 Factors app}
Metodologia per lo sviluppo di applicazioni \glossario{Software-as-a-Service} (SaaS)\\
\url{https://12factor.net/}
\clearpage
