\letteraGlossario{U}

\definizione{UI}
UI (dall'inglese \virgolette{\textit{User Interface}}, in italiano \virgolette{Interfaccia Utente}), è ciò che si frappone tra una macchina e un utente, consentendo l'interazione tra i due.\\
In ambito informatico sono tipicamente utilizzati i seguenti tipi di interfaccia utente:
\begin{itemize}
  \item CLI: Command Line Interface o Interfaccia a riga di comando;
  \item GUI: Graphical User Interface o Interfaccia grafica (GUI).
\end{itemize}


\definizione{UML}
Acronimo per Unified Modeling Language, è un linguaggio visuale di modellazione e specifica basato sul paradigma object-oriented.

\definizione{UNIX}
UNIX è un sistema operativo portabile per computer inizialmente sviluppato da un gruppo di ricerca dei laboratori AT\&T e Bell Laboratories. \`{E} la base di partenza di \glossario{Linux}.\\
I sistemi Unix-like, detti anche di tipo Unix, compatibili con Unix o *nix, sono sistemi operativi aderenti in larga parte agli standard derivati da Unix, tra cui la Single UNIX Specification e POSIX.

\definizione{UTF-8}
UTF-8, ovvero Unicode Transformation Format 8 bit, è una codifica dei caratteri Unicode in sequenze di lunghezza variabile di byte.
\clearpage
