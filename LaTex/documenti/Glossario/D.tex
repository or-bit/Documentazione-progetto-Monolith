\letteraGlossario{D}

\definizione{Database}
Database o \virgolette{base di dati} (a volte abbreviato con la sigla DB dall'inglese data base) indica un insieme di dati, omogeneo per contenuti e per formato, memorizzati in un elaboratore elettronico e interrogabili via terminale utilizzando le chiavi di accesso previste.

\definizione{DBMS}
DBMS, ovvero \virgolette{\textit{Database Management System}} (traducibile grossolanamente in \virgolette{Sistema di gestione di basi di dati}), è un sistema software progettato per consentire la creazione, la manipolazione e l'interrogazione efficiente di database.
In base a come è possibile gestire logicamente i dati in un DBMS si identificano diversi modelli di struttura dei dati; i più frequentemente utilizzati sono:
\begin{itemize}
    \item RDBMS o \textit{Relational} DBMS: i dati vengono organizzati in liste di informazioni chiamati tuple e possono essere correlati tra loro attraverso relazioni;
    \item \textit{document-oriented} DBMS: i dati vengono organizzati come se fossero collezioni di dati semi-strutturati: in base al contesto le informazioni salvate nel Database possono essere strutturate oppure no. In questo tipo di DBMS non c'è separazione tra le informazioni e come esse vengono memorizzate.
\end{itemize}

\definizione{Design Pattern}
Concetto che può essere definito come una soluzione progettuale generale ad un problema ricorrente. Si tratta di una descrizione o modello logico da applicare per la risoluzione di un problema che può presentarsi in diverse situazioni durante le fasi di progettazione e sviluppo del software, ancor prima della definizione dell'algoritmo risolutivo della parte computazionale.

\definizione{Debugging}
Attività che consiste nell'individuazione da parte del programmatore della porzione di software affetta da errore (\glossario{bug}) rilevata nei software a seguito dell'utilizzo del programma.

\definizione{Discord}
Discord è un’applicazione di chat vocale online che offre dei vantaggi come la bassa latenza, server gratis per gli utenti e infrastrutture per server dedicati.\\
\url{https://discordapp.com/}

\definizione{Direttore}
Il direttore è il tipo di utente dell’applicazione che sfrutta quest’ultima per controllare e gestire il flusso di ordinazioni, il magazzino e le consegne.

\definizione{Driver}
Componente attiva fittizia per pilotare i test. Controlla l’esecuzione di procedure che non costituiscano il main di un programma.

\definizione{Docker}
Docker è un progetto \glossario{open source} che automatizza lo sviluppo di applicazioni all'interno di container software, fornendo un'astrazione aggiuntiva grazie alla virtualizzazione a livello di sistema operativo di \glossario{Linux}. Docker utilizza le funzionalità di isolamento delle risorse del kernel \glossario{Linux} per consentire a "container" indipendenti di coesistere sulla stessa istanza di \glossario{Linux}, evitando l'installazione e la manutenzione di una macchina virtuale.\\
\url{https://www.docker.com/}

\definizione{DOM}
Document Object Model (DOM), letteralmente modello a oggetti del documento, è una forma di rappresentazione dei documenti strutturati come modello orientato agli oggetti.
DOM è lo standard ufficiale del \glossario{W3C} per la rappresentazione di documenti strutturati in maniera da essere neutrali sia per la lingua che per la piattaforma. DOM è inoltre la base per una vasta gamma di interfacce di programmazione delle applicazioni; alcune di esse sono standardizzate dal W3C.\\
\url{https://www.w3.org/DOM/}

\definizione{DPI}
Dots Per Inch, letteralmente punti per pollice, è una misura di densità dei punti d'inchiostro su una linea immaginaria lunga un pollice (circa 2.54 cm). Viene usata principalmente per indicare la qualità di stampa di documenti e immagini in formato digitale.
\clearpage
