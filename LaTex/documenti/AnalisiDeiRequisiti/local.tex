%%%%%%%%%%%%%%
%  COSTANTI  %
%%%%%%%%%%%%%%

% In questa prima parte vanno definite le 'costanti' utilizzate soltanto da questo documento.
% Devono iniziare con una lettera maiuscola per distinguersi dalle funzioni.

\newcommand{\DocTitle}{Analisi dei Requisiti}
\newcommand{\DocVersion}{\VersioneAR{}}

\newcommand{\DocRedazione}{\bea{} \\ \alice{} \\ \lorenzo{}}
\newcommand{\DocVerifica}{\mattia{} \\ \tommy{}}
\newcommand{\DocApprovazione}{\alice{}}

\newcommand{\DocUso}{Esterno}
\newcommand{\DocDistribuzione}{
	\Committente{} \\
	Gruppo \GroupName{} \\
	\Proponente{}
}

% La descrizione del documento
\newcommand{\DocDescription}{
Documento relativo all'analisi dei requisiti e dei casi d'uso del progetto \ProjectName{} fatta dal gruppo \GroupName{}.
 }

%%%%%%%%%%%%%%
%  FUNZIONI  %
%%%%%%%%%%%%%%

% In questa seconda parte vanno definite le 'funzioni' utilizzate soltanto da questo documento.

%Counter per casi d'uso
\newcounter{usecase}
\newcounter{usecasefiglio}[usecase]
\newcounter{ucff}[usecasefiglio]
\newcounter{alternativi}

%\newcommand*{\UC}{\stepcounter{usecase} UC\theusecase}
%\newcommand{\UCF}{\stepcounter{usecasefiglio} UC\theusecase.\theusecasefiglio}
\newcommand{\UCCaption}{UC\theusecase{}}
\newcommand{\UCFCaption}{UC\theusecase.\theusecasefiglio{}}
\newcommand{\UCFFCaption}{UC\theusecase.\theusecasefiglio.\theucff{}}
\newcommand{\UCCCaption}{
	\ifisfiglio
		UC\theusecase.\theusecasefiglio.\thealternativi{}
	\else
		UC\theusecase.\thealternativi{}
	\fi
}

%Comandi per creare sezioni per i casi d'uso, evitando gli errori di compilazione
\newif\ifisfirst
\newif\ifisfiglio

\makeatletter
\newcommand*{\textlabel}[2]{%
	\edef\@currentlabel{#1}% Set target label
	\phantomsection% Correct hyper reference link
	\label{#2}% store label
}
\makeatother

\newcommand{\UC}[2]{
	\refstepcounter{usecase}
	\isfirsttrue
	\textlabel{UC\theusecase{}}{#2}
	\subsubsection{UC\theusecase{} #1} 
}
\newcommand{\UCF}[2]{
	\refstepcounter{usecasefiglio} 
	\isfirsttrue 
	\textlabel{UC\theusecase.\theusecasefiglio{}}{#2}
	\paragraph{UC\theusecase.\theusecasefiglio{} #1} 
	\mbox{}\\
}
\newcommand{\UCFF}[2]{
	\refstepcounter{ucff} 
	\isfirsttrue 
	\textlabel{UC\theusecase.\theusecasefiglio.\theucff{}}{#2}
	\subparagraph{UC\theusecase.\theusecasefiglio.\theucff{} #1}
	\mbox{}\\
}
\newcommand{\UCC}[3][1]{
	\ifnum#1=1
		\ifisfirst
			\setcounter{alternativi}{1}
			\global\isfirstfalse
			\global\isfigliofalse
			\refstepcounter{usecase}
			\textlabel{UC\theusecase.\thealternativi{}}{#3}
			\subsubsection{UC\theusecase.\thealternativi{} #2}
		\else
			\refstepcounter{alternativi}
			\textlabel{UC\theusecase.\thealternativi{}}{#3}
			\subsubsection{UC\theusecase.\thealternativi{} #2}
		\fi
	\else
		\ifnum#1=2
			\ifisfirst
				\setcounter{alternativi}{1}
				\global\isfirstfalse
				\global\isfigliotrue
				\refstepcounter{usecasefiglio}
				\textlabel{UC\theusecase.\theusecasefiglio.\thealternativi{}}{#3}
				\paragraph{UC\theusecase.\theusecasefiglio.\thealternativi{} #2}
				\mbox{}\\
			\else
				\refstepcounter{alternativi}
				\textlabel{UC\theusecase.\theusecasefiglio.\thealternativi{}}{#3}
				\paragraph{UC\theusecase.\theusecasefiglio.\thealternativi{} #2}
				\mbox{}\\
			\fi
		\fi
	\fi
}


%Counter per requisiti
\newcounter{requisito}
%Contatori per riepilogo
%funzionali
\newcounter{totObF}
\newcounter{totOpF}
\newcounter{totDF}
%di qualità
\newcounter{totObQ}
\newcounter{totOpQ}
\newcounter{totDQ}
%di vincolo
\newcounter{totObV}
\newcounter{totOpV}
\newcounter{totDV}
%prestazionali
\newcounter{totObP}
\newcounter{totOpP}
\newcounter{totDP}

% Comandi per creare i requisiti
\newcommand*{\req}[2]{
	\refstepcounter{requisito}
	\textlabel{R#1#2\therequisito}{R#1#2\therequisito}
	R#1#2\therequisito
}
% obbligatori
\newcommand*{\RequisitoObF}{\stepcounter{totObF} \req{1}{F}}
\newcommand*{\RequisitoObQ}{\stepcounter{totObQ} \req{1}{Q}}
\newcommand*{\RequisitoObV}{\stepcounter{totObV} \req{1}{V}}
\newcommand*{\RequisitoObP}{\stepcounter{totObP} \req{1}{P}}
% desiderabili
\newcommand*{\RequisitoDF}{\stepcounter{totDF} \req{2}{F}}
\newcommand*{\RequisitoDQ}{\stepcounter{totDQ} \req{2}{Q}}
\newcommand*{\RequisitoDV}{\stepcounter{totDV} \req{2}{V}}
\newcommand*{\RequisitoDP}{\stepcounter{totDP} \req{2}{P}}
% opzionali
\newcommand*{\RequisitoOpF}{\stepcounter{totOpF} \req{3}{F}}
\newcommand*{\RequisitoOpQ}{\stepcounter{totOpQ} \req{3}{Q}}
\newcommand*{\RequisitoOpV}{\stepcounter{totOpV} \req{3}{V}}
\newcommand*{\RequisitoOpP}{\stepcounter{totOpP} \req{3}{P}}
