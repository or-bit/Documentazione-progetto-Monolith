\section{La strategia di gestione della qualità nel dettaglio}

\subsection{Risorse}
Le risorse umane vengono descritte nelle \NormeDiProgetto{}. I ruoli con maggiori responsabilità sono il \Responsabile{} e il \Verificatore{}, soprattutto per l'attività di \VV{}.
Le risorse tecnologiche corrispondono a tutti gli strumenti software e hardware utilizzati per la verifica sui processi e prodotti.

\subsection{Obiettivi per le metriche}\label{sec:metriche}
Per poter descrivere il processo di verifica è necessario che esso sia quantificabile. Questo avviene tramite misure basate su metriche prestabilite consultabili nelle \NormeDiProgetto{}.

\subsubsection{Metriche per la documentazione}\label{sec:metriche_documentazione}
\paragraph{Indice Gulpease}\mbox{}\\
Per garantire la leggibilità della documentazione in lingua italiana si fa riferimento ai seguenti punteggi minimi e ottimali selezionati per raggiungere un grado di comprensibilità adeguato.
\begin{itemize}
	\item range ottimale: 50-100;
	\item range di accettazione: 40-100.
\end{itemize} 

\paragraph{Indice Gunning Fog}\mbox{}\\
Per garantire la leggibilità della documentazione in lingua inglese sono stati selezionati i seguenti punteggi minimi e ottimali dell'indice Gunning Fog.
\begin{itemize}
	\item range ottimale: 14-17;
	\item range di accettazione: 12-17.
\end{itemize}

\subsubsection{Metriche di processo}\label{sec:metriche_processo}

\paragraph{Metriche per la progettazione}\mbox{}\\
Si concentrano sulle caratteristiche dell'architettura ad alto livello. Si basano sull'analisi di modelli di progetto nei quali sono evidenziati i moduli di sistema e i dati scambiati.

\subparagraph{Accoppiamento}\mbox{}\\
Per determinare un livello di accoppiamento accettabile il numero di chiamate reciproche fra classi viene racchiuso nei range di ottimalità e accettazione seguenti.

Fan-In:
\begin{itemize}
	\item range ottimale: 2 o superiore;
	\item range di accettazione 0 o superiore.
\end{itemize}

Fan-Out:
\begin{itemize}
	\item range ottimale: 0-1;
	\item range di accettazione 0-5.
\end{itemize}

\subparagraph{Complessità ciclomatica}\mbox{}\\
Per mantenere sotto controllo la complessità del software prodotto è calcolata la complessità ciclomatica del codice. 
Sono dichiarati come accettabili i seguenti range di complessità per i risultati ottenuti dal calcolo in modo da massimizzarne la manutenibilità e semplicità di testing: 
\begin{itemize}
	\item range 1-10 riguardano programmi semplici con poco rischio d'errore;
	\item range 11-20 adatti a programmi complessi ma a rischio moderato.
\end{itemize}

\paragraph{Metriche per la codifica}\mbox{}

\subparagraph{Numero di parametri per metodo}\mbox{}\\
I limiti imposti per il numero di parametri per metodo sono i seguenti:
\begin{itemize}
	\item range ottimale: 0-4;
	\item range di accettazione: 0-7.
\end{itemize}

\subparagraph{Metriche di Halstead}\mbox{}\\
La difficoltà di scrittura e comprensione del programma calcolato con le metriche di Halstead per funzione:
\begin{itemize}
	\item range ottimale: 0-15;
	\item range di accettazione 0-25.
\end{itemize}

Lo sforzo complessivo di scrittura del programma è calcolato con le metriche di Halstead per funzione:
\begin{itemize}
	\item range ottimale: 0-300;
	\item range di accettazione: 0-400.
\end{itemize}

Il volume del programma è calcolato con le metriche di Halstead per funzione:
\begin{itemize}
	\item range ottimale: 20-100;
	\item range di accettazione: 20-1500.
\end{itemize}

\subparagraph{Core size}\mbox{}\\
Per evitare l'eccessiva interdipendenza tra i file sono stati selezionati i seguenti limiti per il core size:
\begin{itemize}
	\item range ottimale: 0-40;
	\item range di accettazione: 0-50.
\end{itemize}

\subparagraph{Indice di manutenibilità}\mbox{}\\
L'indice di manutenibilità indica approssimativamente la difficoltà di mantenere il prodotto.
\begin{itemize}
	\item range ottimale: 120-171;
	\item range di accettazione: 90-171.
\end{itemize}

\paragraph{Metriche per la verifica}\mbox{}
\subparagraph{Code coverage}\mbox{}\\
La copertura in percentuale dei test effettuati è individuata nel modo seguente:
\begin{itemize}
	\item range ottimale: 60-100;
	\item range di accettazione: 40-100.
\end{itemize}

\subparagraph{Modified condition/decision coverage (MC/DC)}\mbox{}\\
E' raggiunta ottenendo risultati accettabili sia in \textit{branch coverage} che in \textit{function coverage}.

Branch Coverage in percentuale:
\begin{itemize}
	\item range ottimale: 80-100;
	\item range di accettazione: 70-100.
\end{itemize}

Function Coverage in percentuale:
\begin{itemize}
	\item range ottimale: 80-100;
	\item range di accettazione: 70-100.
\end{itemize}

\subparagraph{Test eseguiti}\mbox{}\\
\begin{longtable}[h]{|l|c|c|}
	\hline \multirow{2}{*}{\textbf{Tipo di test}} & \multicolumn{2}{c|}{Range} \\ \cline{2-3}
	& \multicolumn{1}{c|}{\textbf{Ottimale}} & \multicolumn{1}{c|}{\textbf{Accettazione}} \\ \hline
	\endfirsthead
	\hline Unità & 100 & 85-100 \\
	\hline Integrazione & 70-100 & 60-100 \\
	\hline Sistema & 80-100 & 70-100 \\
	\hline Validazione & 100 & 90-100 \\
	\hline
	\caption{Percentuale di test eseguiti}
\end{longtable}

\subparagraph{Test superati}\mbox{}\\
Percentuale di test superati:
\begin{itemize}
	\item range ottimale: 100;
	\item range di accettazione: 85-100.
\end{itemize}

\subparagraph{Metriche di gestione degli errori}\mbox{}\\
\begin{longtable}{|P{6cm}|c|c|P{4cm}|}
	\hline 
	\multicolumn{1}{|c|}{\textbf{Errore}} & \multicolumn{1}{c|}{\textbf{Criticità}} & \multicolumn{1}{c|}{\textbf{Priorità}} & \multicolumn{1}{c|}{\textbf{Modalità}} \\ \hline
	\endfirsthead
	
	\hline 
	\multicolumn{1}{|c|}{\textbf{Errore}} & \multicolumn{1}{c|}{\textbf{Criticità}} & \multicolumn{1}{c|}{\textbf{Priorità}} & \multicolumn{1}{c|}{\textbf{Modalità}} \\ \hline 
	\endhead
	
	\hline \multicolumn{4}{|r|}{\ToBeContinued} \\ \hline
	\endfoot
	
	\hline
	\endlastfoot
	
	\hline Errore ortografico o di formattazione & Bassa & Bassa & Correzione immediata \\
	
	\hline Errore sistematico & Media & Media & Segnalazione \\
	
	\hline Errore compilazione documento & Alta & Alta & Correzione immediata o segnalazione \\
	
	\hline Indici di leggibilità bassi\linebreak(Gulpease <40, Gunning Fog <12 ) & Media & Media & Segnalazione \\
	
	\hline Errore concettuale & Alta & Alta & Segnalazione \\
	
	\hline Errore di progettazione & Alta & Alta & Segnalazione \\
	
	\hline Errore \glossario{UML} & Bassa & Bassa & Correzione immediata o segnalazione \\
	
	\hline Errore compilazione codice & Alta & Alta & Correzione immediata o segnalazione \\
	
	\hline Errore rispetto a norme di codifica & Media & Media & Segnalazione \\
	
	\hline Errore tracciamento & Alta & Alta & Segnalazione \\
	
	\hline
	\caption{Metriche di gestione degli errori}
\end{longtable}

\subsubsection{Metriche di prodotto}
Hanno l'obiettivo di misurare la qualità del prodotto software nelle sue caratteristiche fisiche quali dimensioni, funzionabilità, manutenibilità e usabilità.

\paragraph{Functional Size Measurement}\mbox{}\\
Copertura in percentuale dei requisiti funzionali implementati:
\begin{itemize}
	\item range ottimale: 100;
	\item range di accettazione: 80-100.
\end{itemize}

\paragraph{Accuratezza rispetto alle attese}\mbox{}\\
Percentuale di risultati che rispettano le attese:
\begin{itemize}
	\item range ottimale: 100;
	\item range di accettazione: 80-100.
\end{itemize}

\paragraph{Fallimento dei test}\mbox{}\\
Percentuale di test conclusi con fallimento:
\begin{itemize}
	\item range ottimale: 0;
	\item range di accettazione: 0-15.
\end{itemize}

\paragraph{Gestione delle operazioni non permesse}\mbox{}\\
Percentuale di funzionalità in grado di gestire gli errori per operazioni non permesse:
\begin{itemize}
	\item range ottimale: 100;
	\item range di accettazione: 80-100.
\end{itemize}

