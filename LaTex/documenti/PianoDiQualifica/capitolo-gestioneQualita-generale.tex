\section{Visione generale della strategia di gestione della qualità}
\subsection{Obiettivi di qualità}
Il gruppo si prefigge di realizzare un prodotto conforme rispetto a quanto stabilito nel \PianoDiQualifica{} e \AnalisiDeiRequisiti{}, raggiungendo con la massima efficacia gli obiettivi attesi e completando con la massima efficienza le attività richieste per farlo.\\
Per assicurare il progresso in modo corretto vengono attuati processi di verifica al metodo di lavoro, per accertare che sia svolto nel rispetto delle convenzioni e delle procedure stabilite. Viene inoltre verificato il software prodotto per assicurare l'avanzamento nel ciclo di vita di un prodotto corretto.\\
Affinché il sistema prodotto sia conforme a quanto stabilito, vengono attuati processi di validazione.\\
Vengono utilizzati strumenti di valutazione della qualità dei processi per attuare controlli e attivare un miglioramento continuo di essi.

\subsubsection{Qualità di processo}
Per definire e controllare i processi si è deciso di basarsi sulle norme \textbf{\glossario{ISO} 9000} descritte in \sezione{sec:iso9000} del presente documento.\\
Fra gli standard di valutazione viene adottato \textbf{ISO/\glossario{IEC} 15504}, denominato \textbf{SPICE} (Software Process Improvement and Capability dEtermination).\\
SPICE fornisce gli strumenti necessari a valutare la qualità e l'adeguatezza dei processi.\\
Il modello viene descritto approfonditamente in \appendice{app:spice}.\\
Per utilizzare questo strumento è necessario dotarsi di una disciplina di miglioramento dei processi, realizzata attraverso il \glossario{ciclo di Deming}, definito in \appendice{app:pdca}.

\paragraph{Standard ISO 9000 per la qualità di processo}\label{sec:iso9000} \mbox{}\\
Gli standard ISO 9000 e 9001 per la qualità di processo sono basati su norme e regole raccolte in un sistema di sette principi per la gestione della qualità.
\begin{itemize}
	\item \textbf{Customer focus}: l'impegnarsi al soddisfacimento dei bisogni del cliente e la previsione delle loro future necessità è di fondamentale importanza; ciò aumenta la soddisfazione del cliente con conseguente aumento della fedeltà da parte del consumatore e della reputazione aziendale, utili per aumentare la base clientelare. L'obiettivo di questo principio è l'aumento del mercato attraverso l'incremento qualitativo del rapporto clientelare. 
	\item \textbf{Leadership}: un sistema di ruoli è importante per impostare una direzione comune per il raggiungimento degli obiettivi. La comunicazione e l'assegnazione dei compiti sono i mezzi di coordinazione principale del team per realizzare il proprio coordinamento e quello delle risorse impiegate.
	\item \textbf{Engagement of people}: competenza del personale ad ogni livello e promozione della collaborazione.
	\item \textbf{Process Approach}: per capire la gestione di qualità va compreso che i processi produttivi sono parte di un sistema. Vanno quindi pensate le interdipendenze fra essi e pianificato il loro consumo di risorse. Deve essere compiuto un monitoraggio dei rischi e l'analisi delle performance di processo.
	\item \textbf{Improvement}: per attuare il miglioramento è necessario pianificare le attività e le risorse impiegate. Vanno poi eseguiti e analizzati i processi ed i loro output per applicarne correzioni e migliorie.
	\item \textbf{Evidence-based decision making}: predisporre indicatori per effettuare misurazioni sui dati disponibili rende più probabile prendere decisioni corrette.
	\item \textbf{Relationship management}: il mantenimento dei rapporti con l'esterno è fondamentale per mantenere attivi i processi e reagire a cambiamenti nella catena di fornitura.
\end{itemize}

Vengono ora elencati gli obiettivi che il gruppo si prefigge di raggiungere:
\begin{itemize}
	\item i documenti prodotti devono essere leggibili e comprensibili;
	\item i documenti prodotti devono essere aggiornati;
	\item le componenti progettate devono essere associate ai requisiti;
	\item le componenti progettate devono garantire il maggior riuso possibile;
	\item deve essere garantito un basso accoppiamento ed un'alta coesione;
	\item la progettazione deve agevolare le attività di codifica e i test rispettando la massima precisione possibile;
	\item la progettazione di dettaglio deve definire in modo chiaro l'architettura a basso livello del prodotto software;
	\item il codice prodotto deve essere facilmente comprensibile e garantire una complessità tale da risultare facilmente testabile;
	\item il flusso di esecuzione del codice deve garantire una complessità tale da equilibrare manutenibilità ed efficienza;
	\item il codice deve essere facilmente manutenibile;
	\item non devono esistere variabili inutilizzate;
	\item i test devono essere svolti su un sistema verificato e ben integrato;
	\item il sistema deve implementare i requisiti obbligatori individuati;
	\item l'analisi dinamica deve coprire la maggior parte delle casistiche di esecuzione;
	\item l'analisi dinamica deve coprire la maggior parte delle funzioni del codice;
	\item gli errori devono essere segnalati e corretti secondo quanto stabilito nelle \NormeDiProgetto{}.
\end{itemize}

\paragraph{Procedure di controllo di qualità di processo}\mbox{}\\
Per garantire la qualità dei processi viene applicato il principio \textbf{PDCA} (Plan Do Check Act), descritto in \appendice{app:pdca}.\\ 
L'obiettivo che si vuole raggiungere è il continuo miglioramento della qualità dei processi, e pertanto dei prodotti.\\
Per un controllo completo della qualità, è necessaria una regolamentazione dei processi: essi devono essere pianificati in modo dettagliato, le risorse devono essere assegnate ad essi ed entrambi devono essere sottoposti ad un controllo continuativo.\\
Le metriche di quantificazione della qualità dei processi vengono descritte in \NormeDiProgetto{} e gli obiettivi per le metriche in \sezione{sec:metriche} del presente documento.

\subsubsection{Qualità di prodotto}
Il gruppo si prefigge di realizzare un prodotto che raggiunga la qualità adatta al soddisfacimento degli obiettivi di progetto sanciti per il suo funzionamento.\\
Per compiere ciò si adotta lo standard \textbf{ISO/IEC 9126} per la valutazione della qualità del software. Esso viene descritto approfonditamente in \appendice{app:iso9126} del presente documento.\\
Gli obiettivi che il gruppo si impegna a raggiungere sono:
\begin{itemize}
	\item \textbf{adeguatezza}: le funzionalità fornite sono conformi alle aspettative;
	\item \textbf{accuratezza}: il prodotto fornisce i risultati attesi;
	\item \textbf{maturità}: sono evitati malfunzionamenti, operazioni non consentite e restituzione di risultati errati;
	\item \textbf{tolleranza agli errori}: gli errori sono previsti e gestiti per evitare malfunzionamenti;
\end{itemize}


\paragraph{Procedure di controllo di qualità di prodotto}\mbox{}\\
Per garantire la qualità di prodotto vengono applicate le attività di Quality Assurance, \VV{}.
\begin{itemize}
	\item \textbf{Quality Assurance}: garantisce il raggiungimento degli obiettivi di qualità, tramite analisi statica e dinamica;
	\item \textbf{Verifica}: determina la correttezza dei risultati. \`{E} un’attività continuativa durante lo svolgimento del progetto. Ad ogni verifica vengono riportati i risultati in \appendice{app:resoconto_verifica};
	\item \textbf{Validazione}: corrisponde alla verifica e collaudo finale del prodotto e ne determina il corretto funzionamento rispetto ai requisiti.
\end{itemize}

\subsection{Scadenze temporali}
Le scadenze temporali sono fissate nel \PianoDiProgetto{}, così come le metodologie da seguire per individuare e gestire i rischi.\\
Per evitare errori di impostazione dei documenti, errori tecnici o di contenuto non pertinente ad un documento, si è scelto di far precedere alla redazione dei documenti la stesura della struttura degli stessi.

\subsection{Modalità di verifica}
Sono state individuate varie modalità di verifica al fine di garantire una copertura continua per le varie attività.
\begin{itemize}
	\item \textbf{Travis CI}: vengono eseguiti test sulla corretta compilazione dei documenti e del codice, ad ogni push sul repository, attraverso il sistema di integrazione continua Travis CI. In caso la compilazione non vada a buon fine viene segnalato un errore e il commit non viene integrato nella release.
	\item \textbf{Script}: sono stati creati degli script per automatizzare il controllo lessicografico dei documenti. Viene così garantita la segnalazione di termini scritti con lettere sbagliate, invertite o mancanti, favorendo il lavoro dei \Verificatori{}. Questo tipo di verifica va eseguito dopo ogni modifica \virgolette{importante}, quali il completamento di una sezione o di grandi parti di testo.
	\item \textbf{\Verificatore{}}: è compito del \Verificatore{} eseguire i test che non possono essere svolti in maniera automatica. È altresì compito suo verificare che i risultati dei test effettuati in modo automatico siano corretti. Questo tipo di attività va svolta secondo quanto pianificato nel \PianoDiProgetto{}.
\end{itemize}
