\section{Visione generale della strategia di gestione della qualità}
\subsection{Obiettivi di qualità}
Il gruppo si prefigge di realizzare un prodotto conforme rispetto a quanto stabilito nel \PianoDiQualifica{} e \AnalisiDeiRequisiti{}, raggiungendo con la massima efficacia gli obiettivi attesi e completando con la massima efficienza le attività richieste per farlo.\\
Per assicurare il progresso in modo corretto vengono attuati processi di verifica al metodo di lavoro, per accertare che sia svolto nel rispetto delle convenzioni e delle procedure stabilite. Viene inoltre verificato il software prodotto per assicurare l'avanzamento nel ciclo di vita di un prodotto corretto.\\
Affinché il sistema prodotto sia conforme a quanto stabilito, vengono attuati processi di validazione.\\
Vengono utilizzati strumenti di valutazione della qualità dei processi per attuare controlli e attivare un miglioramento continuo di essi.

\subsubsection{Qualità di processo}
Per definire e controllare i processi si è deciso di basarsi sulle norme \textbf{\glossario{ISO} 9000} descritte in \sezione{sec:iso9000} del presente documento.\\
Fra gli standard di valutazione viene adottato \textbf{\glossario{ISO}/\glossario{IEC} 15504}, denominato \textbf{SPICE} (Software Process Improvement and Capability dEtermination).\\
SPICE fornisce gli strumenti necessari a valutare la qualità e l'adeguatezza dei processi.\\
Il modello viene descritto approfonditamente in \appendice{app:spice}.\\
Per utilizzare questo strumento è necessario dotarsi di una disciplina di miglioramento dei processi, realizzata attraverso il \glossario{ciclo di Deming}, definito in \appendice{app:pdca}.

\paragraph{Standard ISO 9000 per la qualità di processo}\label{sec:iso9000} \mbox{}\\
Gli standard \glossario{ISO} 9000 e 9001 per la qualità di processo sono basati su norme e regole raccolte in un sistema di sette principi per la gestione della qualità.
\begin{itemize}
	\item \textbf{Costumer focus}: l'impegnarsi al soddisfacimento dei bisogni del cliente e la previsione delle loro future necessità è di fondamentale importanza; ciò aumenta la soddisfazione del cliente con conseguente aumento della fedeltà da parte del consumatore e della reputazione aziendale, utili per aumentare la base clientelare. L'obiettivo di questo principio è l'aumento del mercato attraverso l'incremento qualitativo del rapporto clientelare. 
	\item \textbf{Leadership}: un sistema di ruoli è importante per impostare una direzione comune per il raggiungimento degli obiettivi. La comunicazione e l'assegnazione dei compiti sono i mezzi di coordinazione principale del team per realizzare il proprio coordinamento e quello delle risorse impiegate.
	\item \textbf{Engagement of people}: competenza del personale ad ogni livello e promozione della collaborazione.
	\item \textbf{Process Approach}: per capire la gestione di qualità va compreso che i processi produttivi sono parte di un sistema. Vanno quindi pensate le interdipendenze fra essi e pianificato il loro consumo di risorse. Deve essere compiuto un monitoraggio dei rischi e l'analisi delle performance di processo.
	\item \textbf{Improvement}: per attuare il miglioramento è necessario pianificare le attività e le risorse impiegate. Vanno poi eseguiti e analizzati i processi ed i loro output per applicarne correzioni e migliorie.
	\item \textbf{Evidence-based decision making}: predisporre indicatori per effettuare misurazioni sui dati disponibili rende più probabile prendere decisioni corrette.
	\item \textbf{Relationship management}: il mantenimento dei rapporti con l'esterno è fondamentale per mantenere attivi i processi e reagire a cambiamenti nella catena di fornitura.
\end{itemize}

\paragraph{Procedure di controllo di qualità di processo}\mbox{}\\
Per garantire la qualità dei processi viene applicato il principio \textbf{PDCA} (Plan Do Check Act), descritto in \appendice{app:pdca}.\\ 
L'obiettivo che si vuole raggiungere è il continuo miglioramento della qualità dei processi, e pertanto dei prodotti.\\
Per un controllo completo della qualità, è necessaria una regolamentazione dei processi: essi devono essere pianificati in modo dettagliato, le risorse devono essere assegnate ad essi ed entrambi devono essere sottoposti ad un controllo continuativo.\\
Le metriche di quantificazione della qualità dei processi vengono descritte in \sezione{sec:metriche}.

\subsubsection{Qualità di prodotto}
Il gruppo si prefigge di realizzare un prodotto che raggiunga la qualità adatta al soddisfacimento degli obiettivi di progetto sanciti per il suo funzionamento.\\
Per compiere ciò si adotta lo standard \textbf{\glossario{ISO}/\glossario{IEC} 9126} per la valutazione della qualità del software. Esso viene descritto approfonditamente in \appendice{app:iso9126} del presente documento.

\paragraph{Procedure di controllo di qualità di prodotto}\mbox{}\\
Per garantire la qualità di prodotto vengono applicate le attività di Quality Assurance, Verifica e Validazione.
\begin{itemize}
	\item \textbf{Quality Assurance}: garantisce il raggiungimento degli obiettivi di qualità, tramite analisi statica e dinamica;
	\item \textbf{Verifica}: determina la correttezza dei risultati. \`{E} un’attività continuativa durante lo svolgimento del progetto. Ad ogni verifica vengono riportati i risultati in \appendice{app:risultati _verifica};
	\item \textbf{Validazione}: corrisponde alla verifica e collaudo finale del prodotto e ne determina il corretto funzionamento rispetto ai requisiti.
\end{itemize}

\subsection{Scadenze temporali}
Le scadenze temporali sono fissate nel \PianoDiProgetto{}, così come le metodologie da seguire per individuare e correggere gli errori.\\
Per evitare errori di impostazione dei documenti, errori tecnici o di contenuto non pertinente ad un documento, si è scelto di far precedere alla redazione dei documenti la stesura della struttura degli stessi.