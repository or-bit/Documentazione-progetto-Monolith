\subsection{Test di Unità}
\subsubsection{Framework}
\begin{longtable}{|c|P{9cm}|c|}
	\hline \multicolumn{1}{|l|}{\textbf{Codice}} &  \multicolumn{1}{l|}{\textbf{Descrizione}}  \\ 
	\endfirsthead

	\hline \test{U}{L102}\label{tu-genericbubble} & Verifica che la Generic Bubble:
	\begin{itemize}
		\item renderizzi correttamente come componente React;
		\item renderizzi elementi diversi a seconda che sia in vita.
	\end{itemize}
	 \\
%	\hline \test{U}{L103}\label{tu-memory} & Verifica che la Bubble Memory:
%	\begin{itemize}
%		\item sia un oggetto di tipo EventEmitter.
%	\end{itemize}
%	 \\
	\hline \test{U}{L104}\label{tu-database} & Verifica che il DataBase:
	\begin{itemize}
		\item abbia l'URL corretto;
		\item si connetta e riceva l'oggetto in modo corretto;
		\item salvi gli oggetti nella collezione nel modo corretto.
	\end{itemize}
	 \\
	\hline \test{U}{L105}\label{tu-externalapi} & Verifica che ExternalAPI:
	\begin{itemize}
		\item controlli che esista un myurl valido;
		\item riceva un un oggetto JSON valido dall'url.
	\end{itemize}
	 \\
	\hline \test{U}{L106}\label{tu-itemsstore} & Verifica che l'ItemsStore:
	\begin{itemize}
		\item venga creato in modo coretto e sia vuoto;
		\item aggiunga un elemento in modo corretto;
		\item rimuova correttamente l'ultimo elemento.
	\end{itemize}
	 \\
	\hline \test{U}{L107}\label{tu-lifecycle} & Verifica che il LifeCycle:
	\begin{itemize}
		\item venga creato in modo coretto;
		\item faccia il reset del timer correttamente.
	\end{itemize}
	 \\
	\hline \test{U}{L108}\label{tu-regex} & Verifica che la classe MatchRegularExpr:
	\begin{itemize}
		\item venga istanziata in modo coretto;
		\item utilizzi le funzioni di match in modo corretto;
		\item il risultato del match sia corretto.
	\end{itemize}
	 \\
	\hline \test{U}{L109}\label{tu-barchart} & Verifica che il grafico a istogramma:
	\begin{itemize}
		\item renderizzi il corretto numero di barre a seconda dei dati immessi;
	\end{itemize}
	 \\
	\hline \test{U}{L110}\label{tu-button} & Verifica che il componente Button:
	\begin{itemize}
		\item renderizzi correttamente;
		\item esegua correttamente la funzione chiamata cliccando sul componente.
	\end{itemize}
	 \\
	\hline \test{U}{L111}\label{tu-checkbox} & Verifica che il il componente CheckBox:
	\begin{itemize}
		\item renderizzi correttamente;
		\item cambi lo stato quando viene spuntato.
	\end{itemize}
	 \\
	\hline \test{U}{L112}\label{tu-checkboxgroup} & Verifica che il componente CheckBoxGroup: renderizzi correttamente  \\
	\hline \test{U}{L113}\label{tu-image} & Verifica che il componente Image renderizzi correttamente	 \\
	\hline \test{U}{L114}\label{tu-inputfile} & Verifica che l'InputFile sia sia formato in modo corretto  \\
	\hline \test{U}{L115} & Verifica che il componente InputText sia formato in modo corretto  \\
	\hline \test{U}{L116} & Verifica che il componente InputText:
	\begin{itemize}
		\item contenga i giusti tag html;
		\item abbia le corrette proprietà.
	\end{itemize}
	 \\
	\hline \test{U}{L117} & Verifica che il componente Label:
	\begin{itemize}
		\item venga renderizzato;
		\item riceva il valore da renderizzare correttamente.
	\end{itemize}
	 \\
	\hline \test{U}{L118} & Verifica che il componente PieChart:
	\begin{itemize}
		\item renderizzi correttamente i settori a seconda dei dati immessi.
	\end{itemize}
	 \\
	\hline \test{U}{L119} & Verifica che il componente RadioButton:
	\begin{itemize}
		\item renderizzi correttamente;
		\item cambi stato quando il suo valore cambi.
	\end{itemize}
	 \\
	\hline \test{U}{L120} & Verifica che il componente RadioButtonGroup:
	\begin{itemize}
		\item venga renderizzato correttamente;
		\item cambi il proprio stato quando cambia il RadioButtonGroup selezionato.
	\end{itemize}
	 \\
	\hline \test{U}{L121} & Verifica che il componente TextEdit:
	\begin{itemize}
		\item venga renderizzato correttamente;
		\item aggiorni il suo stato quando cambia il valore.
	\end{itemize}
	 \\
	\hline \test{U}{L122}\label{tu-textview} & Verifica che il TextView:
	\begin{itemize}
		\item venga renderizzato;
		\item riceva il valore da renderizzare correttamente.
	\end{itemize}
	 \\
	\hline \test{U}{L123} & Verifica che il componente GUI:
	\begin{itemize}
		\item renderizzi correttamente;
		\item renderizzi correttamente il componente che si vuole creare tramite esso.
	\end{itemize}
	 \\
	\hline \test{U}{L124} & Verifica che il componente GUIContainer venga renderizzato correttamente  \\
	\hline
	\caption{Test di unità per il framework}
\end{longtable}

\subsubsection{To-do List}
\begin{longtable}{|c|P{9cm}|c|}
	\hline \multicolumn{1}{|l|}{\textbf{Codice}} &  \multicolumn{1}{l|}{\textbf{Descrizione}}  \\ 
	\endfirsthead
	\hline \test{U}{L125} & Verifica che il Model:
	\begin{itemize}
		\item crei correttamente un elemento della lista;
		\item crei correttamente la lista;
		\item aggiorni lo stato degli elementi quando gli viene notificato.
	\end{itemize}
	 \\
	\hline \test{U}{L126} & Verifica che il Controller renderizzi correttamente  \\
	\hline \test{U}{L127} & Verifica che la View renderizzi correttamente  \\
	\hline
	\caption{Test di unità per la To-do list}
\end{longtable}

\subsubsection{\DemoName{}}
\begin{longtable}{|c|P{9cm}|c|}
	\hline \multicolumn{1}{|l}{\textbf{Codice}} &  \multicolumn{1}{|l|}{\textbf{Descrizione}}  \\ 
	\endfirsthead
	\hline \test{U}{L128} & Verifica che il server:
	\begin{itemize}
		\item venga creato correttamente;
		\item ascolti alla porta assegnata;
		\item ritorni un socket che non sia nullo.
	\end{itemize}
	 \\
	\hline\test{U}{L129}\label{tu-adminhandler} & Verifica che l'handler dell'admin:
	\begin{itemize}
		\item si connetta correttamente alla porta;
		\item ottenga il menu;
		\item aggiunga un piatto al menu;
		\item modifichi un piatto dal menu;
		\item rimuova un piatto dal menu;
		\item ottenga tutte le ordinazioni;
		\item ottenga le ordinazioni attive;
		\item ottenga le ordinazioni completate;
		\item rimuova un'ordinazione.
	\end{itemize}
	  \\
	 \hline\test{U}{L130}\label{tu-clienthandler} & Verifica che l'handler del client:
	 \begin{itemize}
	 	\item si connette correttamente alla porta;
	 	\item ottenga il menu;
	 	\item crei una nuova ordinazione;
	 	\item ottenga i dati relativo al proprio ordine;
	 	\item ottenga l'id relativo al proprio ordine;
	 	\item riceva una notifica quando l'ordine è completato;
	 	\item riceva un errore quando non viene trovato l'ordine relativo al proprio id;
	 	\item riottenga lo stato del proprio ordine nel caso di una disconnessione;
	 	\item ottenga la notifica del proprio ordine se viene completato mentre l'utente è disconnesso alla sua riconnessione.
	 \end{itemize}
	  \\
	 \hline\test{U}{L131}\label{tu-chefhandler} & Verifica che l'handler dello Chef:
	 \begin{itemize}
	 	\item si connetta correttamente alla porta;
	 	\item si disconnetta correttamente;
	 	\item comunichi correttamente con il database;
	 	\item controlli che riceva correttamente le ordinazioni attive(anche se non ce ne fossero) quando è nello stato di pronto;
	 	\item attivi correttamente il completamento dell'ordinazione.
	 \end{itemize}
 	 \\
	\hline\test{U}{L132}\label{tu-json} & Verifica che l'oggetto JSON dedicato alla connessione sia formattato in modo corretta  \\
%	\hline \test{U}{L133}\label{tu-coo & Verifica che il CookReducer:
%	\begin{itemize}
%		\item legga che lo stato è indefinito se non ancora istanziato;
%		\item imposti in modo corretto lo stato in assente;
%		\item imposti in modo corretto lo stato in presente.
%	\end{itemize}
%	 \\
	\hline \test{U}{L134} & Verifica che il Menu:
	\begin{itemize}
		\item legga che lo stato è indefinito se non ancora istanziato;
		\item aggiunga correttamente il piatto ed incrementi l'id;
		\item rimuova in modo corretto un piatto dal menu;
		\item modifichi correttamente i dati di un piatto senza modificarne l'id.
	\end{itemize}
	 \\
	\hline \test{U}{L135} & Verifica che OrderContainer:
	\begin{itemize}
		\item aggiunga correttamente un'ordinazione;
		\item cambi in modo corretto lo stato di un'ordinazione completata;
		\item rimuova in modo corretto un'ordinazione.
	\end{itemize}
	 \\
%	\hline\test{U}{L136} & Verifica che l'istanza di un oggetto di tipo Actions sia formattato in modo corretto & Superato \\
%	\hline\test{U}{L137} & Verifica che le istanze di tipo Actions della presenza e dell'assenza del cook sono generate in modo corretto & Superato \\
	\hline\test{U}{L138} & Verifica che le chiamate di aggiunta, modifica e rimozione del menu siano generate in modo corretto \\
	\hline\test{U}{L139} & Verifica che le chiamate di aggiunta, modifica e rimozione dell'ordinazione siano generate in modo corretto \\
	\hline \test{U}{L140} & Verifica che la BubbleCustomer:
	\begin{itemize}
		\item renderizzi correttamente;
		\item aggiunga le pietanze all'ordine;
		\item rimuova pietanze dall'ordine;
		\item invii l'ordine;
		\item effettui una richiesta per visualizzare il menu.
	\end{itemize}
	 \\
	\hline \test{U}{L141} & Verifica che la BubbleManager:
	\begin{itemize}
		\item renderizzi correttamente;
		\item esegua una richiesta per aggiungere un piatto al menu;
		\item esegua una richiesta per rimuovere un piatto dal menu;
		\item esegua una richiesta per modificare un piatto del menu;
		\item esegua una richiesta per eliminare un ordine;
		\item esegua una richiesta per visualizzare il menu.
	\end{itemize}
	\\
		\hline \test{U}{L142} & Verifica che la BubbleChef:
	\begin{itemize}
		\item renderizzi correttamente;
		\item esegua una richiesta per completare un ordine.
	\end{itemize}
	 \\
	\hline
	\caption{Test di unità per l'order gateway}
\end{longtable}

%\subsubsection{Tracciamento metodi - test}
%\begin{longtable}{|m{12cm}|>{\centering}m{2cm}|}
%	\hline \textbf{Metodo} & \textbf{Test} \\
%	\code{}
%	\endfirsthead
%\end{longtable}
