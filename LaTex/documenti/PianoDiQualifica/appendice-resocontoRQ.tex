\subsection{\RQ{}}
Sono stati verificati i documenti prodotti mediante le procedure descritte nelle \NormeDiProgetto{}, nella sezione relativa alle procedure di supporto dei processi di \VV{}.\\
Si è proceduto nel modo seguente:\begin{itemize}
	\item gli errori rilevati con più frequenza sono stati integrati alla lista di controllo, allegata in appendice al documento \NormeDiProgetto{};
	\item sono stati corretti gli errori;
	\item è stato applicato il ciclo PDCA per il miglioramento del processo di verifica;
	\item è stata applicata l'\glossario{inspection} utilizzando la lista di controllo;
	\item sono state calcolate le metriche per i documenti descritte in \NormeDiProgetto{};
	\item è stato verificato il tracciamento delle componenti rispetto ai requisiti;
	\item è stato verificato il tracciamento dei test rispetto alle componenti e ai requisiti.
\end{itemize}
L'avanzamento dei processi è stato verificato sulla base di quanto riportato nelle \NormeDiProgetto{}. Sono quindi state calcolate le metriche di processo e del prodotto disponibile e confrontate con gli obiettivi descritti in \sezione{sec:metriche_processo}.

\subsubsection{Esiti delle verifiche}
\paragraph{Indice di Gulpease}\mbox{}\\
\begin{longtable}{|c|c|c|c|c|}
	\hline \multicolumn{1}{|c|}{\textbf{Documento}} & \multicolumn{1}{c|}{\textbf{Risultato}} & \multicolumn{1}{c|}{\textbf{Accettazione}} & \multicolumn{1}{c|}{\textbf{Ottimalità}} & \multicolumn{1}{c|}{\textbf{Esito}}\\
	\hline 
	\endfirsthead
	
	\hline \multicolumn{1}{|c|}{\textbf{Documento}} & \multicolumn{1}{c|}{\textbf{Risultato}} & \multicolumn{1}{c|}{\textbf{Accettazione}} & \multicolumn{1}{c|}{\textbf{Ottimalità}} & \multicolumn{1}{c|}{\textbf{Esito}}\\
	\hline 
	\endhead
	
	\hline \multicolumn{5}{|r|}{\ToBeContinued} \\ 
	\hline
	\endfoot
	
	\hline
	\endlastfoot
	
	\hline \NormeDiProgetto{} & 79 & 40-100 & 50-100 & Superato\\
	\hline \PianoDiProgetto{} & 80 & 40-100 & 50-100 & Superato \\
	\hline \PianoDiQualifica{} & 80 & 40-100 & 50-100 & Superato \\
	\hline \AnalisiDeiRequisiti{} & 92 & 40-100 & 50-100 & Superato \\
	\hline \Glossario{} & 65 & 40-100 & 50-100 & Superato \\
	\hline \SpecificaTecnica{} & 85 & 40-100 & 50-100 & Superato\\
	\hline \DefinizioneDiProdotto{} & 72 & 40-100 & 50-100 & Superato\\
	\hline Demo User manual & 8.1 & 6-12 & 6-10 & Superato\\
	\hline Framework User manual & 7.6 & 6-12 & 6-10 & Superato\\
	\hline VerbaleInterno25\_04\_2017 & 75 & 40-100 & 50-100 & Superato \\
	\hline VerbaleInterno10\_04\_2017 & 93 & 40-100 & 50-100 & Superato \\
	\hline VerbaleInterno29\_03\_2017 & 85 & 40-100 & 50-100 & Superato \\
	\hline VerbaleEsterno22\_03\_2017 & 68 & 40-100 & 50-100 & Superato \\
	\hline VerbaleInterno2\_03\_2017 & 73 & 40-100 & 50-100 & Superato \\
	\hline VerbaleInterno23\_02\_2017 & 83 & 40-100 & 50-100 & Superato \\
	\hline VerbaleInterno30\_01\_2017 & 85 & 40-100 & 50-100 & Superato \\
	\hline VerbaleInterno30\_12\_2016 & 84 & 40-100 & 50-100 & Superato \\
	\hline VerbaleEsterno23\_12\_2016 & 75 & 40-100 & 50-100 & Superato \\
	\hline VerbaleInterno22\_12\_2016 & 85 & 40-100 & 50-100 & Superato \\
	\hline VerbaleInterno16\_12\_2016 & 83 & 40-100 & 50-100 & Superato \\
	\hline VerbaleInterno9\_12\_2016 & 80 & 40-100 & 50-100 & Superato \\
	\hline VerbaleEsterno8\_12\_2016 & 79 & 40-100 & 50-100 & Superato \\
	\hline
	\caption{Valori indice di Gulpease/Gunnig Fog - Revisione di Progettazione}
\end{longtable}


\paragraph{Fallimento dei test}\mbox{}\\
La percentuale di test falliti sulla compilazione dei documenti ottenuta è del 13\%, un valore che ricade nel range di accettazione di questa metrica secondo quanto imposto da questo documento. 
\\Il fallimento rilevato in questa metrica nella revisione precedente ha portato ad una maggiore attenzione all'evitare errori nella compilazione dei documenti in \LaTeX, riducendo il numero di test falliti.

\paragraph{Metriche di progettazione e di codifica}\mbox{}\\
Vengono ora presentati i risultati ottenuti dalle misurazioni relativamente alle metriche di progettazione e di codifica:
\begin{longtable}{|m{5cm}|c|c|c|c|}
	\hline \multicolumn{1}{|c|}{\textbf{Metrica}} & \multicolumn{1}{c|}{\textbf{Risultato}} & \multicolumn{1}{c|}{\textbf{Accettazione}} & \multicolumn{1}{c|}{\textbf{Ottimalità}} & \multicolumn{1}{c|}{\textbf{Esito}}\\
	\hline 
	\endfirsthead
	
	\hline \multicolumn{1}{|c|}{\textbf{Metrica}} & \multicolumn{1}{c|}{\textbf{Risultato}} & \multicolumn{1}{c|}{\textbf{Accettazione}} & \multicolumn{1}{c|}{\textbf{Ottimalità}} & \multicolumn{1}{c|}{\textbf{Esito}}\\
	\hline 
	\endhead
	
	\hline \multicolumn{5}{|r|}{\ToBeContinued} \\ 
	\hline
	\endfoot
	
	\hline
	\endlastfoot
	
	\hline Fan-in medio & 1.20 & >=0 & >=2 & Superato \\
	\hline Fan-out medio & 3 & 0-5 & 0-1 & Superato \\
	\hline Complessità ciclomatica & 2.45 & 1-15 & 1-10 & Superato \\
	\hline Parametri per metodo & 2.4 & 0-7 & 0-4 & Superato \\
	\hline Halstead: difficoltà & 13 & 0-25 & 0-15 & Superato \\
	\hline Halstead: sforzo complessivo & 314.5 & 0-400 & 0-300 & Superato\\
	\hline Halstead: volume & 1053 & 20-1500 & 20-1000 & Superato \\
	\hline Core size & 22 & 0-50 & 0-40 & Superato \\
	\hline Indice di manutenibilità & 126 & 90-171 & 120-171 & Superato \\
	\hline Code coverage\footnote{Questa metrica è stata calcolata escludendo le tre classi BubbleCustomer, BubbleChef e BubbleManager} & 91 & 40-100 & 60-100 & Superato \\
	\hline Branch coverage\footnote{Questa metrica è stata calcolata escludendo le tre classi BubbleCustomer, BubbleChef e BubbleManager} & 75 & 70-100 & 80-100 & Superato \\
	\hline Function coverage\footnote{Questa metrica è stata calcolata escludendo le tre classi BubbleCustomer, BubbleChef e BubbleManager} & 87 & 70-100 & 80-100 & Superato \\
	\hline Test di unità eseguiti & 84 & 85-100 & 100 & Non superato \\
	\hline Test di integrazione eseguiti & 50 & 60-100 & 70-100 & Non superato \\
	\hline Test di sistema eseguiti & 63 & 70-100 & 80-100 & Non superato \\
	\hline Test superati & 100 & 85-100 & 100 & Superato \\
	\hline
	\caption{Metriche di processo - \RQ{}}
\end{longtable}

\paragraph{Metriche di prodotto}\mbox{}\\
\begin{longtable}{|m{5cm}|c|c|c|c|}
	\hline \multicolumn{1}{|c|}{\textbf{Metrica}} & \multicolumn{1}{c|}{\textbf{Risultato}} & \multicolumn{1}{c|}{\textbf{Accettazione}} & \multicolumn{1}{c|}{\textbf{Ottimalità}} & \multicolumn{1}{c|}{\textbf{Esito}}\\
	\hline 
	\endfirsthead
	
	\hline \multicolumn{1}{|c|}{\textbf{Metrica}} & \multicolumn{1}{c|}{\textbf{Risultato}} & \multicolumn{1}{c|}{\textbf{Accettazione}} & \multicolumn{1}{c|}{\textbf{Ottimalità}} & \multicolumn{1}{c|}{\textbf{Esito}}\\
	\hline 
	\endhead
	
	\hline \multicolumn{5}{|r|}{\ToBeContinued} \\ 
	\hline
	\endfoot
	
	\hline
	\endlastfoot
	
	\hline Functional size measurement & 72 & 80-100 & 100 & Non superato \\
	\hline Accuratezza rispetto alle attese & 91 & 90-100 & 100 & Superato\\
	\hline Fallimento dei test & 9 & 0-15 & 0 & Superato \\
	\hline Parametri per metodo & 1.2 & 0-7 & 0-4 & Superato \\
	\hline
	\caption{Metriche di prodotto - \RQ{}}
\end{longtable}