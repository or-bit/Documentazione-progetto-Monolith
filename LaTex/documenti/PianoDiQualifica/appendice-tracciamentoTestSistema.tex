\subsection{Test di sistema}

%\subsubsection{Framework}
%
%\begin{longtable}{|P{5cm}|P{5cm}|P{5cm}|}
%	\hline \textbf{Codice} & \textbf{Descrizione} & \textbf{Requisito} \\
%	\endfirshead
%	
%	\hline \test{S}{L} & Data una GUI con più elementi e un elemento funzionale, simulare un input utente e far inviare un segnale dalla GUI alla bubble, seguendo tutto il percorso, per controllare che ciò che la GUI ritorna sia corretto & \\
%	\hline \test{S}{L} & Data una GUI con più elementi e un elemento funzionale, simulare un input utente e far inviare un segnale non corretto dalla GUI alla bubble, seguendo tutto il percorso, per controllare che la GUI ritorni un errore e non si aggiorni le informazioni sbagliate & \\
%	\hline \test{S}{L} & Data una GUI con più elementi e un elemento funzionale, simulare un evento originato dall'elemento funzionale e verificare che questo abbia ripercussioni sulla bubble generica e sulla GUI & \\
%	\hline \test{S}{L} & Data una GUI con un elemento e più elementi funzionali, simulare un input utente e far inviare un segnale dalla GUI alla bubble, tracciando tutto il percorso, per controllare che ciò che la GUI ritorna sia corretto & \\
%	\hline \test{S}{L} & Data una GUI con un elemento e più elementi funzionali, simulare un input utente e far inviare un segnale non corretto dalla GUI alla bubble, seguendo tutto il percorso, per controllare che la GUI ritorni un errore e non si aggiorni le informazioni sbagliate & \\
%	\hline \test{S}{L} & Data una GUI con un elemento e più elementi funzionali, simulare un evento originato dall'elemento funzionale e verificare che questo abbia ripercussioni sulla bubble generica e sulla GUI & \\
%	\hline \test{S}{L} & Data una GUI con più elementi e più elementi funzionali, simulare un input utente e far inviare un segnale dalla GUI alla bubble, tracciando tutto il percorso, per controllare che ciò che la GUI ritorna sia corretto & \\
%	\hline \test{S}{L} & Data una GUI con più elementi e più elementi funzionali, simulare un input utente e far inviare un segnale non corretto dalla GUI alla bubble, seguendo tutto il percorso, per controllare che la GUI ritorni un errore e non si aggiorni le informazioni sbagliate & \\
%	\hline \test{S}{L} & Data una GUI con un elemento e più elementi funzionali, simulare un evento originato dall'elemento funzionale e verificare che questo abbia ripercussioni sulla bubble generica e sulla GUI & \\
%	\hline
%	\caption{Test di sistema per il framework}
%\end{longtable}

\subsubsection{Bubble To-do list}

\begin{longtable}{|c|m{7cm}|c|m{3cm}|}
	\hline \multicolumn{1}{|l|}{\textbf{Codice}} &  \multicolumn{1}{l|}{\textbf{Descrizione}} & \multicolumn{1}{l|}{\textbf{Requisito}}  \\ 
	\endfirsthead
	\hline \test{S}{L45} & Verificare che la bubble To-do list permetta all'utente la creazione di liste & \ref{AdR-L17}  \\
	\hline \test{S}{L46} & Verificare che l'utente possa invocare il comando di creazione della to-do list & \ref{AdR-L71}  \\
	\hline \test{S}{L47} & Verificare che la bubble carichi un form per l'inserimento delle informazioni & \ref{AdR-L95}   \\
	\hline \test{S}{L48} & Verificare che l'utente possa inserire le informazioni per la creazione della to-do list all’interno del form & \ref{AdR-L72}  \\
	\hline \test{S}{L49} & Verificare che l'utente possa confermare le informazioni inserite visualizzandole in un riepilogo & \ref{AdR-L73}  \\
	\hline \test{S}{L50} & Verificare che la bubble To-do list permetta l’inserimento di nuovi elementi nella lista tramite un form & \ref{AdR-L18}  \\
	\hline \test{S}{L51} & Verificare che una volta aggiunti nuovi elementi la bubble li mostri nella To-do list & \ref{AdR-L74}  \\
	\hline \test{S}{L52} & Verificare che la bubble To-do list permetta di segnare come completati gli elementi della lista & \ref{AdR-L19}  \\
	\hline \test{S}{L53} & Verificare che una volta segnato come completato un elemento della To-do list questo venga disabilitato e reso non modificabile & \ref{AdR-L75}  \\
	\hline \test{S}{L54} & Verificare che la To-do list permetta di settare un reminder come notifica statica & \ref{AdR-L20}  \\
	\hline \test{S}{L55} & Verificare che la notifica statica venga visualizzata una volta scaduto l'intervallo temporale impostato & \ref{AdR-L76}  \\
	\hline
	\caption{Test di sistema per la bubble To-do list}
\end{longtable}

\subsubsection{Bubble \& eat}

\begin{longtable}{|c|m{7cm}|c|m{3cm}|}
	\hline \multicolumn{1}{|l|}{\textbf{Codice}} & \multicolumn{1}{l|}{\textbf{Descrizione}} & \multicolumn{1}{l|}{\textbf{Requisito}} \\  
	\endfirsthead
	
	\hline \test{S}{L56} & Verificare che la bubble per la ristorazione si interfacci con un database per il salvataggio delle informazioni & \ref{AdR-L21}  \\
	\hline \test{S}{L57} & Verificare che la bubble per la ristorazione si interfacci con un database per il recupero delle informazioni & \ref{AdR-L61}  \\
	\hline \test{S}{L58} & Verificare che la bubble per la ristorazione si interfacci con un database per aggiornare informazioni & \ref{AdR-L62}  \\	
	\hline \test{S}{L59} & Verificare che i Customer per poter utilizzare la bubble possano inserire le proprie informazioni & \ref{AdR-L22}  \\
	\hline \test{S}{L60} & Verificare che la bubble mostri un form per l'inserimento delle informazioni personali dell'utente & \ref{AdR-L96}  \\
	\hline \test{S}{L61} & Verificare che l'utente possa inserire le informazioni nel form & \ref{AdR-L97}  \\
	\hline \test{S}{L62} & Verificare che l'utente possa confermare le informazioni inserite visualizzandole nella bubble & \ref{AdR-L98}  \\
	\hline \test{S}{L63} & Verificare che i Customer possano consultare il menu del ristorante, visualizzandolo nella bubble & \ref{AdR-L23}  \\
	\hline \test{S}{L64} & Verificare che i Customer visualizzino i prezzi delle pietanze nel menu & \ref{AdR-L68}  \\
	\hline \test{S}{L65} & Verificare che i Customer possano selezionare cibi e relative quantità per effettuare un ordine & \ref{AdR-L24}  \\
	\hline \test{S}{L66} & Verificare che la quantità di default per un cibo non selezionato sia 0 & \ref{AdR-L69}  \\
	\hline \test{S}{L67} & Verificare che la quantità di default per un cibo selezionato sia 1 & \ref{AdR-L77}  \\
	\hline \test{S}{L68} & Verificare che non possa essere incrementata la quantità per un cibo non selezionato & \ref{AdR-L70}  \\
	\hline \test{S}{L69} & Verificare che l'utente possa confermare di aver concluso la modifica cliccando su un pulsante di conferma & \ref{AdR-L99}  \\
	\hline \test{S}{L70} & Verificare che la bubble mostri all'utente un resoconto dell'ordine prima della sua conferma definitiva & \ref{AdR-L100}  \\
	\hline \test{S}{L71} & Verificare che l'utente Chef possa visualizzare la lista di piatti da preparare & \ref{AdR-L25}  \\
	\hline \test{S}{L72} & Verificare che l'utente Chef possa marcare come completati i piatti già preparati tra quelli dalla lista di piatti da preparare  & \ref{AdR-L26} \\
	\hline \test{S}{L73} & Verificare che i piatti spuntati dallo Chef vengono eliminati dalla lista dei piatti da preparare & \ref{AdR-L78}  \\
	\hline \test{S}{L74} & Verificare che l'utente Purchasing Manager possa visualizzare la lista degli acquisti da effettuare & \ref{AdR-L27}  \\
	\hline \test{S}{L75} & Verificare che l'utente Purchasing Manager possa spuntare i prodotti che ha acquistato dalla lista acquisti & \ref{AdR-L28} \\
	\hline \test{S}{L76} & Verificare che l'utente Manager possa visualizzare il menu del ristorante all’interno della bubble & \ref{AdR-L104}   \\
	\hline \test{S}{L77} & Verificare che l'utente Manager possa cambiare il menu del ristorante & \ref{AdR-L105}  \\
	\hline \test{S}{L78} & Verificare che l'utente Manager possa modificare le voci del menu ed i relativi prezzi all’interno della bubble & \ref{AdR-L29}  \\
	\hline \test{S}{L79} & Verificare che l'utente Manager possa aggiungere le voci del menu con relativi prezzi tramite un form & \ref{AdR-L79}  \\
	\hline \test{S}{L80} & Verificare che l'utente Manager possa selezionare l'opzione per aggiungere elemento & \ref{AdR-L108}  \\
	\hline \test{S}{L81} & Verificare che l'utente Manager possa rimuovere le voci del menu & \ref{AdR-L80}  \\
	\hline \test{S}{L82} & Verificare che l'utente Manager possa selezionare le voci del menu & \ref{AdR-L106}  \\
	\hline \test{S}{L83} & Verificare che l'utente Manager possa visualizzare la lista dei piatti da preparare all’interno della bubble & \ref{AdR-L30}  \\
	\hline \test{S}{L84} & Verificare che l'utente Manager possa visualizzare la lista degli acquisti da effettuare all’interno della bubble & \ref{AdR-L31}  \\
	\hline \test{S}{L85} & Verificare che l'utente Manager possa eliminare voci dalla lista dei piatti da preparare & \ref{AdR-L32}  \\
	\hline \test{S}{L86} & Verificare che l'utente Manager possa selezionare ordini nella lista dei piatti da preparare & \ref{AdR-L101}  \\
	\hline \test{S}{L87} & Verificare che l'utente Manager possa selezionare l'opzione per rimuovere gli elementi selezionati & \ref{AdR-L102}  \\
	\hline \test{S}{L88} & Verificare che l'utente Manager possa selezionare l'opzione per modificare gli elementi selezionati & \ref{AdR-L107}  \\
	\hline \test{S}{L89} & Verificare che l'utente Manager possa confermare di voler effettuare l'operazione tramite un pulsante & \ref{AdR-L103}  \\
	\hline \test{S}{L90} & Verificare che l'utente Manager possa aggiungere ingredienti alla lista degli acquisti da effettuare & \ref{AdR-L53} \\
	\hline \test{S}{L91} & Verificare che l'utente Manager possa eliminare ingredienti dalla lista degli acquisti da effettuare & \ref{AdR-L65}  \\
	\hline \test{S}{L92} & Verificare che l'utente Manager possa selezionare elementi dalla lista degli acquisti da effettuare & \ref{AdR-L109}  \\
	\hline \test{S}{L93} & Verificare che l'utente Deliveryman possa visualizzare la lista delle consegne da effettuare all’interno della bubble & \ref{AdR-L50}   \\
	\hline \test{S}{L94} & Verificare che l'utente Deliveryman possa selezionare la consegna che intende effettuare & \ref{AdR-L51}  \\
	\hline \test{S}{L95} & Verificare che l'utente Deliveryman possa confermare la consegna che ha effettuato & \ref{AdR-L52}  \\
	\hline
	\caption{Test di sistema per la \DemoName{}}
\end{longtable}
