\section{Riunione}
\subsection{Ordine del Giorno}
\begin{itemize}
	\item Dimostrazione demo;
	\item resoconto funzionalità implementate;
	\item varie ed eventuali.
\end{itemize}

\subsection{Dimostrazione demo}
Come introduzione alla riunione viene effettuata una breve dimostrazione della demo, così come è stata presentata in sede di revisione.
Premessa è che la demo è ristretta ai ruoli \textit{Customer}, \textit{Chef} e \textit{Manager}, primari per la demo stessa. Il proponente concorda su questa scelta e non ha ulteriori richieste per gli altri ruoli non implementati.

\subsubsection{Heroku}
La demo viene presentata tramite screen sharing, in quanto il caricamento dell'applicazione su \glossario{Heroku} è ancora in sviluppo. Sono stati riscontrati alcuni problemi nel caricare in Heroku sia il front end che il back end in un unico contenitore. È stata effettuata una seconda prova caricando le due componenti in contenitori distinti e mettendo in comunicazione le due parti, ma Heroku chiude le comunicazioni se non riceve segnali entro un certo periodo di tempo, rendendo l'applicazione inutilizzabile.\\
VE\_2017-06-11\_D1: Si è deciso pertanto di ritornare alla prima versione.

\subsubsection{Presentazione grafica delle bubble}
Il proponente chiede a titolo informativo la motivazione per cui i vari ordini si trovino all'interno della stessa bubble e non su diversi messaggi. Questa disposizione è stata pensata per raccogliere le informazioni ed evitare la dispersione all'interno della chat. Viene inoltre proposto l'utilizzo dei webhook per la condivisione delle informazioni tra la web app indipendente e Rocket.Chat.\\
VE\_2017-06-11\_D2: La variante proposta sarà presa in considerazione, ma non ne viene assicurata l'implementazione.

\subsection{Descrizione progetto}
Viene quindi effettuata, su richiesta di \Proponente{}, una descrizione del progetto ad alto livello, specificando la suddivisione delle componenti software utilizzate. Il punto focale della descrizione è stata la suddivisione globale in tre parti:
\begin{itemize}
	\item web app esterna con un database dedicato;
	\item \glossario{bubble}, attualmente inserita come frame;
	\item \glossario{Rocket.Chat}, a cui viene integrata la bubble.
\end{itemize}

\subsection{Autenticazione}
L'autenticazione avviene in avvio della connessione tra la bubble e il server. La bubble comunica il proprio ruolo al server, che accetterà la richiesta e inizializzerà la connessione sulla base del ruolo comunicato. È già predisposta dunque la base per implementare una modalità di autenticazione più complessa.

\subsection{SDK}
L'\glossario{SDK} è quindi diviso in due parti, front end e back end, utilizzabili anche in modo indipendente. Esso permette di creare le bubble all'interno di Rocket.Chat, dentro al quale un pacchetto intercetta i messaggi e renderizza le bubble.

Lo sviluppatore per creare una bubble può utilizzare le componenti fornitegli, integrando così la sua web app in Rocket.Chat.
Per chiarire meglio questo punto, viene richiesta una piccola demo \virgolette{Hello World} che metta in evidenza l'utilizzo del framework, in modo tale da mostrare la semplicità o complessità di utilizzo del framework da parte di un utente esterno al progetto.

\subsection{Conclusione}
Il proponente rimane in attesa dell'esempio \virgolette{Hello World}, che il gruppo si impegna a produrre entro la fine della giornata.
Complessivamente \Proponente{} si dichiara soddisfatta della visione con cui è stato sviluppato il progetto, anche sulla base della suddivisione dei componenti e sui loro nomi, che si allineano con il suo modo di pensare la struttura di un'applicazione web. 

\clearpage
