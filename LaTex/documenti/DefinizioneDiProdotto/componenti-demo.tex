\subsection{\DemoName}\setclass{BubbleAndEat}\label{\class}
La bubble \DemoName{} è suddivisa in una parte server, ovvero il package \hyperref[BubbleAndEat::Restaurant::OrderGateway]{\code{BubbleAndEat::Reastau\-rant::OrderGateway}}, e una parte client, ovvero le bubble. Il componente OrderGateway si occupa di gestire tutti i dati base dell'applicazione, mantenendoli aggiornati tra le diverse bubble esistenti. Il diagramma complessivo è presentato in \hyperref[fig:bubbleandeat]{figura \ref{fig:bubbleandeat}}.
%\clearpage
\newgeometry{margin=1.5cm}
\begin{landscape}
	\pagestyle{empty}
	\centering
	\begin{figure}
		\centering
		\includegraphics{./diagrammi/demo/bubbleandeat.png}
		\caption{Diagramma applicazione \DemoName{}}\label{fig:bubbleandeat}
	\end{figure}
\end{landscape}
\restoregeometry
\pagestyle{plain}
\setclass{BubbleAndEat::Restaurant::OrderGateway}
\subsubsection{Server - \class}
\begin{figure}[H]
	\centering
	\includegraphics[width=15cm]{./diagrammi/demo/server/ordergatewaypkg.png}
	\caption{Componente \class}
\end{figure}
La parte server della \DemoName{} è formata da:
\begin{itemize}
\item OrderGateway, la classe principale che amministra tutte le operazioni legate al server stesso;
\item Server, che crea il server vero e proprio;
\item Menu e MenuItem, che rappresentano il menu del ristorante;
\item Il package Order, che si occupa di rappresentare gli ordini;
\item Il package Handler, che gestisce le richieste da parte delle diverse bubble.
\end{itemize}

\setclass{BubbleAndEat::Restaurant::OrderGateway::Server}
\paragraph[::Server]{\class}\mbox{}\\ \label{\class}
\begin{figure}[H]
	\centering
		\includegraphics[width=5cm]{./diagrammi/demo/server/server.png}
	\caption{Classe \class}
\end{figure}
\textbf{Descrizione:}\\
Classe che rappresenta il server.

\textbf{Utilizzo:}\\
Viene utilizzata per inizializzare il server stesso e i socket.

%\textbf{Classi ereditate:}
%\begin{itemize}
%	\item \code{}.
%\end{itemize}
%
%\textbf{Sottoclassi:}
%\begin{itemize}
%	\item \coderef{}.
%\end{itemize}

\textbf{Attributi:}
\begin{itemize}
	\item \field{- app: Express}: applicazione Express;
	\item \field{- server: http.Server}: server http;
	\item \field{- socket: socket.io}: socket sul server;
	\item \field{- handlers: Map<string, Handler>}: oggetto che contiene gli handlers associati ad una chiave letterale.
\end{itemize}

\textbf{Metodi:}
\begin{itemize}
	\item \method{+ Server()}: costruttore di default, inizializza gli attributi;
	\item \method{+ open(port: int): void}: mette in ascolto il server sulla porta indicata:
	\begin{itemize}
		\item \param{port: int}: porta da ascoltare;
	\end{itemize}
	\item \method{+ close(): void}: termina il server;
	\item \method{+ getSocket(): socket.io}: getter per \texttt{socket};
\end{itemize}

\setclass{BubbleAndEat::Restaurant::OrderGateway::OrderGateway}
\paragraph[::OrderGateway]{\class}\mbox{}\\ \label{\class}
\begin{figure}[H]
	\centering
	\includegraphics[width=8cm]{./diagrammi/demo/server/ordergateway.png}
	\caption{Classe \class}
\end{figure}
\textbf{Descrizione:}\\
La classe OrderGateway rappresenta un gateway per le comunicazioni tra le bubble e il server. In particolare si occupa di gestire e indirizzare le diverse attività della demo. OrderGateway reindirizza le diverse connessioni all'handler della bubble corretta e salva le varie informazioni necessarie sul DB, come lo stato dell'applicazione e gli ordini completati. È compito di questa classe anche quello di avviare il server.

\textbf{Utilizzo:}\\
Viene utilizzata per inizializzare le connessioni e indirizzare le comunicazioni alle giuste bubble attraverso appositi handler. Si occupa anche di avviare il server e salvare le informazioni importanti sul databasae.

%\textbf{Classi ereditate:}
%\begin{itemize}
%	\item \code{}.
%\end{itemize}
%
%\textbf{Sottoclassi:}
%\begin{itemize}
%	\item \coderef{}.
%\end{itemize}

\textbf{Attributi:}
\begin{itemize}
	\item \field{- mongoClient: MongoClient}: database MongoDB;
	\item \field{- mongoUrl: string}: URI del database;
	\item \field{- store: Redux::Store}: oggetto contenente lo stato dell'applicazione;
	\item \field{- handlers: map<string, Handler>}: oggetto che contiene gli handlers associati ad una chiave letterale.
\end{itemize}

\textbf{Metodi:}
\begin{itemize}
	\item \method{+ OrderGateway()}: costruttore, inizializza lo stato dell'applicazione recuperandolo, se possibile, dal database;
	\item \method{+ saveStateToDB(): void}: effettua la connessione al database e salva lo stato dell'applicazione;
	\item \method{+ addCompletedOrderToDB(ordination: Order): void}: aggiunge l'ordine alla collezione degli ordini completati sul database:
	\begin{itemize}
		\item \param{ordination: Order}: l'ordine completato da aggiungere;
	\end{itemize}
	\item \method{+ run(): void}: avvia il server;
	\item \method{+ handle(socket: socket.io, authData: string): void}: gestisce le connessioni indirizzandole al corretto handler.
\end{itemize}

\setclass{BubbleAndEat::Restaurant::OrderGateway::Handlers}
\paragraph[::Handlers]{\class}\mbox{}\\ \label{\class}
\begin{figure}[H]
	\centering
		\includegraphics[width=15cm]{./diagrammi/demo/server/handlerpkg.png}
	\caption{Classe \class}
\end{figure}
Questo componente raggruppa gli handler per le varie bubble. Gli handler hanno lo scopo di gestire le richieste delle bubble. Sono presenti ManagerHandler, CustomerHandler e ChefHandler.

\setclass{BubbleAndEat::Restaurant::OrderGateway::Handlers::Handler}
\subparagraph[::Handler]{\class}\mbox{}\\ \label{\class}
\begin{figure}[H]
	\centering
		\includegraphics[width=15cm]{./diagrammi/demo/server/handlers/handler.png}
	\caption{Classe \class}
\end{figure}
\textbf{Descrizione:}\\
Classe astratta che fornisce una base comune per gli handlers. In particolare contiene ed offre l'accesso all'elenco degli ordini del ristorante, al socket necessario per le comunicazioni con il server e allo store redux dell'applicazione.

\textbf{Utilizzo:}\\
Viene utilizzata come base per gestire gli handlers tramite polimorfismo.

%\textbf{Classi ereditate:}
%\begin{itemize}
%	\item \code{}.
%\end{itemize}
%
\textbf{Sottoclassi:}
\begin{itemize}
	\item \coderef{BubbleAndEat::Restaurant::OrderGateway::Handlers::ManagerHandler}.
	\item \coderef{BubbleAndEat::Restaurant::OrderGateway::Handlers::ChefHandler}.
	\item \coderef{BubbleAndEat::Restaurant::OrderGateway::Handlers::CustomerHandler}.
\end{itemize}

\textbf{Attributi:}
\begin{itemize}
	\item \field{- orders: OrderContainer}: ordini del ristorante;
	\item \field{- socket: socket.io}: socket per le comunicazioni con il server;
	\item \field{- store: Redux::Store}: store redux dell'appicazione.
\end{itemize}

\textbf{Metodi:}
\begin{itemize}
	\item \method{\# Handler(orders: OrderContainer[], socket: socket.io, store: Redux::Store)}: costruttore, assegna i parametri agli attributi:
	\begin{itemize}
		\item \param{orders: OrderContainer}: contenitore degli ordini;
		\item \param{socket: socket.io}: socket di connessione;
		\item \param{store: Redux::Store}: store dell'applicazione;
	\end{itemize}
	\item \method{+ getOrders(): OrderContainer}: getter per \texttt{orders};
	\item \method{+ getSocket(): socket.io}: getter per \texttt{socket};
	\item \method{+ getStore(): Redux::Store}: getter per \texttt{store};
	\item \method{+ disconnect(): void}: termina la connessione e le comunicazioni del Manager con il server.
\end{itemize}

\setclass{BubbleAndEat::Restaurant::OrderGateway::Handlers::ManagerHandler}
\subparagraph[::ManagerHandler]{\class}\mbox{}\\ \label{\class}
\begin{figure}[H]
	\centering
		\includegraphics[width=15cm]{./diagrammi/demo/server/handlers/managerhandler.png}
	\caption{Classe \class}
\end{figure}
\textbf{Descrizione:}\\
La classe ManagerHandler si occupa di gestire le richieste provenienti dalla bubble ManagerBubble ed eseguire le operazioni necessarie sul server. Per fare questo ManagerHandler possiede un corrispondente dei metodi necessari a svolgere tutte le attività del Manager che operano sul server, i quali vengono invocati al momento opportuno dalle loro controparti della bubble BubbleManager quando le diverse funzionalità vengono utilizzate. 

\textbf{Utilizzo:}\\
Viene utilizzata per gestire l'autenticazione del Manager ed eseguire le operazioni richieste dalla bubble lato client sul server.

\textbf{Classi ereditate:}
\begin{itemize}
	\item \coderef{BubbleAndEat::Restaurant::OrderGateway::Handlers::Handler}.
\end{itemize}
%
%\textbf{Sottoclassi:}
%\begin{itemize}
%	\item \coderef{}.
%\end{itemize}

\textbf{Attributi:}
\begin{itemize}
	\item \field{- menu: Menu}: menu del ristorante.
\end{itemize}

\textbf{Metodi:}
\begin{itemize}
	\item \method{+ ManagerHandler(socket: socket.io, store: Redux::Store)}: costruttore, assegna i parametri agli attributi:
	\begin{itemize}
		\item \param{socket: socket.io}: socket contenente i parametri di connessione;
		\item \param{store: Redux::Store}: store dell'applicazione;
	\end{itemize}
	\item \method{+ showMenu(): Menu}: ritorna il menu del ristorante;
	\item \method{+ addDish(dish: MenuItem): MenuItem}: aggiunge una pietanza al menu:
	\begin{itemize}
		\item \param{dish: MenuItem}: la pietanza da aggiungere;
	\end{itemize}
	\item \method{+ removeDish(id: int): int}: rimuove la pietanza dal menu:
	\begin{itemize}
		\item \param{id: int}: id della pietanza da rimuovere;
	\end{itemize}
	\item \method{+ editDish(id: int, dish: MenuItem): MenuItem}: modifica la pietanza del menu in \texttt{dish}:
	\begin{itemize}
		\item \param{id: int}: id della pietanza da modificare;
		\item \param{dish: MenuItem}: pietanza con dati aggiornati;
	\end{itemize}
	\item \method{+ getAllOrders(): OrderContainer}: ritorna tutti gli ordini presenti nell'applicazione;
	\item \method{+ getActiverOrders(): OrderContainer}: ritorna tutti gli ordini attivi (non completati) dell'applicazione;
	\item \method{+ getCompletedOrders(): OrderContainer}: ritorna tutti gli ordini completati dell'applicazione;
	\item \method{+ deleteOrder(id: int): int}: elimina l'ordine dall'applicazione:
	\begin{itemize}
		\item \param{id: int}: id dell'ordine da eliminare.
	\end{itemize}
\end{itemize}

\setclass{BubbleAndEat::Restaurant::OrderGateway::Handlers::ChefHandler}
\subparagraph[::ChefHandler]{\class}\mbox{}\\ \label{\class}
\begin{figure}[H]
	\centering
		\includegraphics[width=15cm]{./diagrammi/demo/server/handlers/chefhandler.png}
	\caption{Classe \class}
\end{figure}
\textbf{Descrizione:}\\
La classe ChefHandler gestisce le richieste provenienti dalla bubble ChefBubble e le reindirizza al server. Possiede dunque il metodo necessario a notificare lo Chef della presenza di una nuova ordinazione da preparare e quello per registrare il completamento di un ordine.

\textbf{Utilizzo:}\\
Viene utilizzata per gestire l'autenticazione dello Chef, eseguire le operazioni richieste dalla bubble lato client sul server e notificare il client della presenza di nuovi ordini da preparare.

\textbf{Classi ereditate:}
\begin{itemize}
	\item \coderef{BubbleAndEat::OrderGateway::Handlers::Handler}.
\end{itemize}
%
%\textbf{Sottoclassi:}
%\begin{itemize}
%	\item \coderef{}.
%\end{itemize}

%\textbf{Attributi:}
%\begin{itemize}
%\end{itemize}

\textbf{Metodi:}
\begin{itemize}
	\item \method{+ ChefHandler(socket: socket.io, store: Redux::Store, orders: OrderContainer)}: costruttore, assegna i parametri agli attributi e salva nello store la presenza dello Chef:
	\begin{itemize}
		\item \param{socket: socket.io}: socket contenente i parametri di connessione;
		\item \param{store: Redux::Store}: store dell'applicazione;
		\item \param{orders: OrderContainer}: contenitore di ordini nello stato di \virgolette{attivo};
	\end{itemize}
	\item \method{+ notify(): OrderContainer}: notifica lo Chef di una nuova ordinazione da preparare;
	\item \method{+ completeOrder(id: int): void}: salva l'ordine come completato:
	\begin{itemize}
		\item \param{id: int}: id dell'ordine completato.
	\end{itemize}
\end{itemize}


\setclass{BubbleAndEat::Restaurant::OrderGateway::Handlers::CustomerHandler}
\subparagraph[::CustomerHandler]{\class}\mbox{}\\ \label{\class}
\begin{figure}[H]
	\centering
		\includegraphics[width=15cm]{./diagrammi/demo/server/handlers/customerhandler.png}
	\caption{Classe \class}
\end{figure}
\textbf{Descrizione:}\\
La classe CustomerHandler gestisce le richieste provenienti dalla bubble CustomerBubble e le reindirizza al server. Fornisce i metodi necessari a recuperare il menu, effettuare un ordine e controllarne lo stato, che vengono invocati quando viene effettuata la richiesta corrispondente dal client.

\textbf{Utilizzo:}\\
Viene utilizzata per gestire l'autenticazione del Customer ed eseguire le diverse operazioni richieste dalla bubble lato client sul server. Si occupa anche di tenere aggiornate le informazioni sullo stato dell'ordine.

\textbf{Classi ereditate:}
\begin{itemize}
	\item \coderef{BubbleAndEat::Restaurant::OrderGateway::Handlers::Handler}.
\end{itemize}
%
%\textbf{Sottoclassi:}
%\begin{itemize}
%	\item \coderef{}.
%\end{itemize}

%\textbf{Attributi:}
%\begin{itemize}
%\end{itemize}

\textbf{Metodi:}
\begin{itemize}
	\item \method{+ CustomerHandler(socket: socket.io, store: Redux::Store, orders: OrderContainer)}: costruttore, assegna i parametri agli attributi:
	\begin{itemize}
		\item \param{socket: socket.io}: socket contenente i parametri di connessione;
		\item \param{store: Redux::Store}: store dell'applicazione;
		\item \param{orders: OrderContainer}: ordini effettuati dal cliente;
	\end{itemize}
	\item \method{+ getOrderStatus(id: int): string}: ritorna lo stato dell'ordine:
	\begin{itemize}
		\item \param{id: int}: id dell'ordine di cui si vuole conoscere lo stato;
	\end{itemize}
	\item \method{+ order(order: Order): id}: processa l'ordine confermato dal Customer e gli assegna un id, che viene ritornato al Customer cosicché egli possa tracciare lo stato dell'ordine stesso:
	\begin{itemize}
		\item \param{order: Order}: l'ordine confermato e inviato;
	\end{itemize}
	\item \method{+ getMenu(): Menu}: recupera il menu dallo store e lo ritorna.
\end{itemize}

\setclass{BubbleAndEat::Restaurant::OrderGateway::MenuItem}
\paragraph[::MenuItem]{\class}\mbox{}\\ \label{\class}
\begin{figure}[H]
	\centering
	\includegraphics[width=10cm]{./diagrammi/demo/server/menuitem.png}
	\caption{Classe \class}
\end{figure}
\textbf{Descrizione:}\\
Classe che rappresenta una singola voce del menu.

\textbf{Utilizzo:}\\
Viene utilizzata per creare le singole pietanze da mostrare nel menu e da inserire negli ordini.

%\textbf{Classi ereditate:}
%\begin{itemize}
%	\item \code{}.
%\end{itemize}
%
%\textbf{Sottoclassi:}
%\begin{itemize}
%	\item \coderef{}.
%\end{itemize}

\textbf{Attributi:}
\begin{itemize}
	\item \field{- id: int}: numero identificativo della pietanza del menu;
	\item \field{- name: string}: nome della pietanza;
	\item \field{- price: double}: prezzo della pietanza;
	\item \field{- description: string}: descrizione della pietanza.
\end{itemize}

\textbf{Metodi:}
\begin{itemize}
	\item \method{+ MenuItem(id: int, name: string, price: double, description: string)}: costruttore, assegna i parametri ai corrispondenti attributi:
	\begin{itemize}
		\item \param{id: int}: id della pietanza;
		\item \param{name: string}: nome della pietanza;
		\item \param{price: double}: prezzo della pietanza;
		\item \param{description: string}: descrizione della pietanza;
	\end{itemize}
	\item \method{+ getId(): int} getter per \texttt{id};
	\item \method{+ setId(id: int): void}: setter per \texttt{id};
	\item \method{+ getName(): string} getter per \texttt{name};
	\item \method{+ setName(name: string): void}: setter per \texttt{name};
	\item \method{+ getPrice(): double} getter per \texttt{price};
	\item \method{+ setPrice(price: double): void}: setter per \texttt{price};
	\item \method{+ getDescription(): string} getter per \texttt{description};
	\item \method{+ setDesctiption(description: string): void}: setter per \texttt{description};
\end{itemize}

\setclass{BubbleAndEat::Restaurant::OrderGateway::Menu}
\paragraph[::Menu]{\class}\mbox{}\\ \label{\class}
\begin{figure}[H]
	\centering
	\includegraphics[width=8cm]{./diagrammi/demo/server/menu.png}
	\caption{Classe \class}
\end{figure}
\textbf{Descrizione:}\\
Classe che rappresenta il menu.

\textbf{Utilizzo:}\\
Viene utilizzata per raccogliere e gestire le pietanze disponibili nel ristorante.

%\textbf{Classi ereditate:}
%\begin{itemize}
%	\item \code{}.
%\end{itemize}
%
%\textbf{Sottoclassi:}
%\begin{itemize}
%	\item \coderef{}.
%\end{itemize}

\textbf{Attributi:}
\begin{itemize}
	\item \field{- dishes: MenuItem[] = []}: array contenete le pietanze del menu;
	\item \field{- date: Date = new Date()}: data di creazione del menu;
	\item \field{- nextId: int = 0}: numero di pietanze inserite e id per la nuova pietanza da inserire.
\end{itemize}

\textbf{Metodi:}
\begin{itemize}
	\item \method{+ Menu()}: costruttore di default;
	\item \method{+ Menu(dishes: MenuItem[])}: costruttore, inizializza l'array delle pietanze:
	\begin{itemize}
		\item \param{dishes: MenuItem[]}: array di pietanze;
	\end{itemize}
	\item \method{+ getMenu(): MenuItem[]} getter per \texttt{dishes};
	\item \method{+ addDish(dish: MenuItem): Menu}: permette di aggiungere una pietanza al menu:
	\begin{itemize}
		\item \param{dish: Menuitem}: pietanza da aggiungere
	\end{itemize}
	\item \method{+ removeDish(dishId: int): Menu} rimuove la pietanza con id \texttt{dishId} dal menu:
	\begin{itemize}
		\item \param{dishId: int}: id della pietanza da rimuovere;
	\end{itemize}
	\item \method{+ modifyDish(id: int, dish: MenuItem): Menu}: modifica la pietanza con id \texttt{id} in \texttt{dish}:
	\begin{itemize}
		\item \param{id: int}: id della pietanza da modificare;
		\item \param{dish: MenuItem}: pietanza con informazioni aggiornate.
	\end{itemize}
\end{itemize}

\setclass{BubbleAndEat::Restaurant::OrderGateway::Order}
\paragraph[::Order]{\class}\mbox{}\\ \label{\class}
\begin{figure}[H]
	\centering
	\includegraphics[width=12cm]{./diagrammi/demo/server/orderpkg.png}
	\caption{Componente \class}
\end{figure}

Questo componente ha la funzione di rappresentare gli ordini.

\setclass{BubbleAndEat::Restaurant::OrderGateway::Order::OrderContainer}
\subparagraph[::OrderContainer]{\class}\mbox{}\\ \label{\class}
\begin{figure}[H]
	\centering
	\includegraphics[width=7cm]{./diagrammi/demo/server/order/ordercontainer.png}
	\caption{Classe \class}
\end{figure}
\textbf{Descrizione:}\\
Classe che rappresenta un contenitore per gli ordini.

\textbf{Utilizzo:}\\
Viene utilizzata come contenitore comune degli ordini attivi per tutti gli utenti interni al ristorante.

%\textbf{Classi ereditate:}
%\begin{itemize}
%	\item \code{}.
%\end{itemize}
%
%\textbf{Sottoclassi:}
%\begin{itemize}
%	\item \coderef{}.
%\end{itemize}

\textbf{Attributi:}
\begin{itemize}
	\item \field{- {orders: Order[] = []}}: array contenente gli ordini, inizializzato di default ad un array vuoto.
\end{itemize}

\textbf{Metodi:}
\begin{itemize}
	\item \method{+ process(order: Order): Order[]}: processa l'ordine, lo imposta come attivo e lo aggiunge all'array \texttt{orders}:
	\begin{itemize}
		\item \param{order: Order}: il nuovo ordine da processare;
	\end{itemize}
	\item \method{+ complete(orderId: int): Order}: l'ordine con id \texttt{orderId} viene completato e il suo stato aggiornato;
	\item \method{+ delete(orderId: int): Order[]}: l'ordine con id \texttt{orderId} viene eliminato dalla lista degli ordini attivi;
	\item \method{+ getOrders(): Order[]}: getter per \texttt{orders}.
\end{itemize}

\setclass{BubbleAndEat::Restaurant::OrderGateway::Order::Order}
\subparagraph[::Order]{\class}\mbox{}\\ \label{\class}
\begin{figure}[H]
	\centering
	\includegraphics[width=12cm]{./diagrammi/demo/server/order/order.png}
	\caption{Classe \class}
\end{figure}
\textbf{Descrizione:}\\
Classe che rappresenta un singolo ordine.

\textbf{Utilizzo:}\\
Viene utilizzata per rappresentare i singoli ordini effettuati dai clienti, ovvero l'insieme delle pietanze che un utente ha ordinato.

%\textbf{Classi ereditate:}
%\begin{itemize}
%	\item \code{}.
%\end{itemize}
%
%\textbf{Sottoclassi:}
%\begin{itemize}
%	\item \coderef{}.
%\end{itemize}

\textbf{Attributi:}
\begin{itemize}
	\item \field{- id: int}: numero identificativo dell'ordine;
	\item \field{- client: Client}: cliente che ha effettuato l'ordine;
	\item \field{- dishes: OrderItem[]}: lista dei piatti ordinati;
	\item \field{- state: string}: indica lo stato di avanzamento dell'ordine.
\end{itemize}

\textbf{Metodi:}
\begin{itemize}
	\item \method{+ Order(client: Client, dishes: OrderItem[])}: costruttore, crea un nuovo ordine con cliente e pietanze ordinate:
	\begin{itemize}
		\item \param{client: Client}: cliente che ha effettuato l'ordine;
		\item \param{dishes: MenuItem[]}: pietanze ordinate;
	\end{itemize}
	\item \method{+ setState(state: string): void}: permette di aggiornare lo stato dell'ordine.
\end{itemize}

\setclass{BubbleAndEat::Restaurant::OrderGateway::OrderItem}
\subparagraph[::OrderItem]{\class}\mbox{}\\ \label{\class}
\begin{figure}[H]
	\centering
	\includegraphics[width=12cm]{./diagrammi/demo/server/order/orderitem.png}
	\caption{Classe \class}
\end{figure}
\textbf{Descrizione:}\\
Classe che rappresenta una pietanza dell'ordine.

\textbf{Utilizzo:}\\
Viene utilizzata per memorizzare le quantità per ogni singolo piatto incluso nell'ordine.

%\textbf{Classi ereditate:}
%\begin{itemize}
%	\item \code{}.
%\end{itemize}
%
%\textbf{Sottoclassi:}
%\begin{itemize}
%	\item \coderef{}.
%\end{itemize}

\textbf{Attributi:}
\begin{itemize}
	\item \field{- dish: MenuItem}: piatto selezionato dal menu;
	\item \field{- amount: int = 1}: quantità selezionata (di default vale 1, altrimenti la voce dell'ordine non ha senso di esistere).
\end{itemize}

\textbf{Metodi:}
\begin{itemize}
	\item \method{+ OrderItem(dish: MenuItem)}: costruttore della classe, crea una voce dell'ordine per la pietanza \emph{dish};
	\item \method{+ getDish(): MenuItem}: getter per \texttt{dish};
	\item \method{+ getAmount(): int}: getter per \texttt{amount};
	\item \method{+ setAmount(amount: int): void}: setter per \texttt{amount}.
\end{itemize}
\subsubsection{Client}
La parte client della \DemoName{} è formata sostanzialmente dalle singole bubble, ognuna risiedente nel proprio package utente. Vi è inoltre la classe Client, che viene posta all'interno del componente Customer, data la correlazione con la bubble Customer.
\setclass{BubbleAndEat::Customer}
\paragraph[::Customer]{\class}\mbox{}\\ \label{\class}
\begin{figure}[H]
	\centering
	\includegraphics[width=7cm]{./diagrammi/demo/client/customer.png}
	\caption{Classe \class}
\end{figure}
Questo componente rappresenta il cliente e la sua bubble.

\setclass{BubbleAndEat::Customer::Client}
\subparagraph[::Client]{\class}\mbox{}\\ \label{\class}
\begin{figure}[H]
	\centering
	\includegraphics[width=7cm]{./diagrammi/demo/client/customer/client.png}
	\caption{Classe \class}
\end{figure}
\textbf{Descrizione:}\\
Classe che rappresenta un cliente.

\textbf{Utilizzo:}\\
Viene utilizzata per raccogliere le informazioni dei clienti.

%\textbf{Classi ereditate:}
%\begin{itemize}
%	\item \code{}.
%\end{itemize}
%
%\textbf{Sottoclassi:}
%\begin{itemize}
%	\item \coderef{}.
%\end{itemize}

\textbf{Attributi:}
\begin{itemize}
	\item \field{- name: string}: numero cliente;
	\item \field{- address: string}: indirizzo di spedizione.
\end{itemize}

\textbf{Metodi:}
\begin{itemize}
	\item \method{+ Client(name: string, address: string)}: costruttore, assegna i parametri;
	\begin{itemize}
		\item \param{name: string}: nome cliente;
		\item \param{address: string}: indirizzo di spedizione;
	\end{itemize}
	\item \method{+ getName(): string}: getter per \texttt{name};
	\item \method{+ setName(name: string): void}: setter per \texttt{name};
	\item \method{+ getAddress(): string}: getter per \texttt{address};
	\item \method{+ setAddress(address: string): void}: setter per \texttt{address}.
\end{itemize}

\setclass{BubbleAndEat::Customer::BubbleCustomer}
\paragraph[::BubbleCustomer]{\class}\mbox{}\\ \label{\class}
\begin{figure}[H]
	\centering
	\includegraphics[width=12cm]{./diagrammi/demo/client/customer/bubblecustomer.png}
	\caption{Classe \class}
\end{figure}
\textbf{Descrizione:}\\
La classe BubbleCustomer rappresenta il Customer e tutte le sue funzionalità all'interno dell'applicazione. In particolare dispone di tutti i metodi necessari a svolgere le diverse attività rese disponibili al Customer, come l'effettuare un ordine o il visualizzare il menu. Questa classe si occupa anche di renderizzare la componente di GUI della bubble, e fornisce tutti i metodi necessari a navigarne le diverse pagine. Infine la bubble dispone dei metodi necessari a connettersi e disconnettersi dal socket che la collega al resto dell'applicazione.

\textbf{Utilizzo:}\\
Viene utilizzata per gestire le funzionalità dei clienti e renderizzare la parte grafica della bubble.

\textbf{Classi ereditate:}
\begin{itemize}
	\item \code{React::Component}.
\end{itemize}

%\textbf{Sottoclassi:}
%\begin{itemize}
%	\item \coderef{}.
%\end{itemize}

\textbf{Attributi:}
\begin{itemize}
%	\item \field{- props: Object[]}: array contenente le proprietà degli elementi della lista;
	\item \field{- menu: Menu}: menu del ristorante;
	\item \field{- page: string}: stringa che indica la pagina corrente;
	\item \field{- quantita: int[]}: array delle quantità dei diversi piatti presenti nell'ordinazione;
	\item \field{- order: Order}: oggetto che rappresenta l'ordine;
	\item \field{- orderState: string} stringa che indica lo stato dell'ordine;
	\item \field{- client: Client}: oggetto che rappresenta il cliente;
	\item \field{- total: double}: prezzo totale dell'ordine;
	\item \field{- socket: socket.io}: oggetto tramite il quale la classe gestisce la connessione con le altre componenti dell'applicazione;
	\item \field{- orderId: int}: id dell'ordine del cliente.
\end{itemize}

\textbf{Metodi:}
\begin{itemize}
	\item \method{+ BubbleCustomer(client: Client)}: costruttore, assegna i parametri:
		\begin{itemize}
			\item \param{client: Client}: cliente utilizzatore della bubble;
		\end{itemize}
	\item \method{+ componentDidMount(): void}: viene invocata alla creazione della classe e invoca la funzione connect;
	\item \method{+ componentWillUnmount(): void}: chiama il metodo disconnect alla distruzione della classe;
	\item \method{+ connect(): void}: connette la classe al resto dell'applicazione tramite socket;
	\item \method{+ showMenu(): void}: invia la richiesta di visualizzazione del menu;
	\item \method{+ order(something: Order): void}: invia l'ordine selezionato;
		\begin{itemize}
			\item \param{something: Order} l'ordine selezionato.
		\end{itemize}
	\item \method{+ queryFor(orderId: int): void}: invia la richiesta di informazioni sullo stato dell'ordine selezionato;
		\begin{itemize}
			\item \param{orderId: int}: id che identifica l'ordine.
		\end{itemize}
	\item \method{+ disconnect(): void}: libera le risorse quando viene distrutta la classe;
	\item \method{+ redirectToHome(): void}: renderizza e si sposta sulla pagina home;
	\item \method{+ redirectToMenu(): void}: renderizza e si sposta sulla pagina menu;
	\item \method{+ redirectToNewOrder(): void}: renderizza e si sposta sulla pagina di costruzione dell'ordine;
	\item \method{+ rediretToOrder(): void}: renderizza e si sposta sulla pagina di gestione dell'ordine; 
	\item \method{+ redirectToInfo() :void}: renderizza e si sposta sulla pagina info;
	\item \method{+ reloadTotal(): double}: ricalcola il prezzo totale e ne restituisce il valore;
	\item \method{+ addDishToOrder(i: int): void}: invoca updateAmount per aggiungere il piatto selezionato all'ordine;
		\begin{itemize}
			\item \param{i: int}: intero che identifica il piatto;
		\end{itemize}
	\item \method{+ removeDishToOrder(i: int): void}: invoca updateAmount per rimuovere il piatto selezionato dall'ordine;
		\begin{itemize}
			\item \param{i:int}: intero che identifica il piatto;
		\end{itemize}
	\item \method{+ handleChange(e: string, i: int): void}: gestisce un cambiamento manuale della quantità di un piatto;
		\begin{itemize}
			\item \param{e: Event}: evento;
			\item \param{i: int}: id del piatto.
		\end{itemize}
	\item \method{+ updateAmount(amount: int, i: int): void}: aggiorna le quantità dei diversi piatti;
	\item \method{+ updateClient(e: string, data: string): void}: aggiorna i dati del cliente;
		\begin{itemize}
			\item \param{e: Event}: evento avvenuto, contiene le nuove informazioni da inserire;
			\item \param{data: string}: indica il tipo di informazione modificata;
		\end{itemize}
	\item \method{+ confirmOrder(): void}: effettua l'ordine;
	\item \method{+ aliveRender(): React::Component}: renderizza la bubble nello stato attivo;
	\item \method{+ notAliveRender(): React::Component}: renderizza la bubble nello stato non attivo;
	\item \method{+ render(): React::Component}: renderizza la bubble.
\end{itemize}

\setclass{BubbleAndEat::Restaurant::Chef::BubbleChef}
\paragraph[::Restaurant::Chef::BubbleChef]{\class}\mbox{}\\ \label{\class}
\begin{figure}[H]
	\centering
	\includegraphics[width=12cm]{./diagrammi/demo/client/bubblechef.png}
	\caption{Classe \class}
\end{figure}
\textbf{Descrizione:}\\
La classe BubbleChef rappresenta lo Chef e le sue funzionalità all'interno dell'applicazione. Questa classe dispone dei metodi che svolgono le diverse attività dello Chef, ovvero il recuperare la lista degli ordini e l'indicare il completamento di uno di questi. BubbleChef si occupa anche di renderizzare la parte di GUI della bubble e dispone dei metodi necessari a connettersi e disconnettersi dal resto dell'applicazione tramite socket.

\textbf{Utilizzo:}\\
Viene utilizzata per gestire le funzionalità della bubble Chef e renderizzarne la parte grafica.

\textbf{Classi ereditate:}
\begin{itemize}
	\item \coderef{Framework::Controller::GenericBubble}.
\end{itemize}

%\textbf{Sottoclassi:}
%\begin{itemize}
%	\item \coderef{}.
%\end{itemize}

\textbf{Attributi:}
\begin{itemize}
	%	\item \field{- props: Object[]}: array contenente le proprietà degli elementi della lista;
	\item \field{- orders: OrderContainer}: contenitore delle ordinazioni da completare;
	\item \field{- socket: socket.io}: oggetto tramite il quale la classe gestisce la connessione con le altre componenti dell'applicazione.
\end{itemize}

\textbf{Metodi:}
\begin{itemize}
	\item \method{+ BubbleChef()}: costruttore di default;
	\item \method{+ componentDidMount(): void}: viene invocata alla creazione della classe e invoca la funzione connect;
	\item \method{+ connect(): void}: crea la connessione con il server;
	\item \method{+ fetchOrders(): void}: invia la richiesta di visualizzazione delle ordinazioni;
	\item \method{+ disconnect(): void}: libera le risorse quando viene distrutta la classe;
	\item \method{+ markOrdinationCompleted(id: int): void}: indica come completata un'ordinazione;
		\begin{itemize}
			\item \param{id: int}: identifica l'ordine da completare;
		\end{itemize}
	\item \method{+ aliveRender(): React::Component}: renderizza la bubble nello stato attivo;
	\item \method{+ notAliveRender(): React::Component}: renderizza la bubble nello stato non attivo;
	\item \method{+ render(): React::Component}: renderizza la bubble.
\end{itemize}


\setclass{BubbleAndEat::Restaurant::Manager::BubbleManager}
\paragraph[::Restaurant::Manager::BubbleManager]{\class}\mbox{}\\ \label{\class}
\begin{figure}[H]
	\centering
	\includegraphics[width=14cm]{./diagrammi/demo/client/manager.png}
	\caption{Classe \class}
\end{figure}
\textbf{Descrizione:}\\
Questa classe che rappresenta la bubble del Manager all'interno dell'applicazione. BubbleManager rende disponibili tutti i metodi necessari a svolgere le diverse attività del manager, come modificare il menu e gestire gli ordini, e si occupa di invocarli in base all'interazione con l'utente. La classe renderizza anche la componente di GUI della bubble e dispone di tutti i metodi necessari a navigarne le diverse pagine. BubbleManager mantiene anche una lista sempre aggiornata di tutti gli ordini correntemente attivi e di quelli completati. Infine la classe dispone anche dei metodi necessari a collegarsi e scollegarsi dal resto dell'applicazione tramite il socket.

\textbf{Utilizzo:}\\
Viene utilizzata per fornire all'utente Manager tutte le funzionalità a lui disponibili e per renderizzare la parte grafica della bubble.

\textbf{Classi ereditate:}
\begin{itemize}
	\item \coderef{Framework::Controller::GenericBubble}.
\end{itemize}
%
%\textbf{Sottoclassi:}
%\begin{itemize}
%	\item \coderef{}.
%\end{itemize}

\textbf{Attributi:}
\begin{itemize}
	\item \field{- menu: Menu}: menu del ristorante;
	\item \field{- allOrders: OrderContainer}: contenitore di tutti gli ordini esistenti nell'applicazione;
	\item \field{- completedOrders: OrderContainer}: contenitore di tutti e soli gli ordini completati;
	\item \field{- activeOrders: OrderContainer}: contenitore di tutti e soli gli ordini attivi (non completati);
	\item \field{- page: string}: pagina da renderizzare;
	\item \field{- formDataName: string}: valore contenuto nel campo \virgolette{Name} di creazione e modifica di una pietanza del menu;
	\item \field{- formDataPrice: double}: valore contenuto nel campo \virgolette{Price} di creazione e modifica di una pietanza del menu;
	\item \field{- socket: socket.io}: socket per la connessione al server.
\end{itemize}

\textbf{Metodi:}
\begin{itemize}
	\item \method{+ BubbleManager()}: costruttore di default;
	\item \method{+ componentDidMount(): void}: metodo invocato in automatico al caricamento della bubble, richiama in sè il metodo \texttt{connect};
	\item \method{+ connect(): void}: crea la connessione con il server;
	\item \method{+ showMenu(): void}: invia una richiesta al server per ricevere il menu aggiornato e lo assegna a \texttt{menu};
	\item \method{+ addDish(dish: MenuItem): void}: aggiunge una pietanza al menu e lo segnala al server;
	\item \method{+ removeDish(id: int): void}: rimuove una pietanza dal menu e lo segnala al server;
	\item \method{+ editDish(id: int, newDish: MenuItem): void}: modifica la pietanza con id \texttt{id} e lo segnala al server;
	\item \method{+ fetchAllOrders(): void}: invia una richiesta al server per ricevere il contenitore degli ordini aggiornato e lo assegna a \texttt{allOrders};
	\item \method{+ fetchActiveOrders(): void}: invia una richiesta al server per ricevere il contenitore degli ordini attivi e lo assegna a \texttt{activeOrders};
	\item \method{+ fetchCompletedOrders(): void}: invia una richiesta al server per ricevere il contenitore degli ordini completati e lo assegna a \texttt{completedOrders};
	\item \method{+ deleteOrder(orderId:int): void}: elimina un ordine e lo segnala al server:
	\begin{itemize}
		\item \param{orderId: int}: id dell'ordine da eliminare;
	\end{itemize}
	\item \method{+ disconnect(): void}: chiude la connessione con il server;
	\item \method{- formOnClick(): void}: definisce il comportamento della bubble quando viene azionato il \virgolette{submit} del form di aggiunta/modifica;
	\item \method{+ redirectToFormAdd(): void}: invoca la renderizzazione della pagina di aggiunta di una pietanza al menu;
	\item \method{+ redirectToFormEdit(): void}: invoca la renderizzazione della pagina di modifica di una pietanza al menu;
	\item \method{+ redirectToHome(): void}: invoca la renderizzazione della pagina principale della bubble;
	\item \method{+ redirectToMenu(): void}: invoca la renderizzazione della pagina di visualizzazione del menu;
	\item \method{+ redirectToOrders(): void}: invoca la renderizzazione della pagina di visualizzazione degli ordini;
	\item \method{+ aliveRender(): React::Component}: renderizza la bubble nello stato attivo;
	\item \method{+ notAliveRender(): React::Component}: renderizza la bubble nello stato non attivo;
	\item \method{+ render(): React::Component}: renderizza la bubble.
	
\end{itemize}